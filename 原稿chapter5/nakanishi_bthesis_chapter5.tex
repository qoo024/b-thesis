\documentclass[uplatex, dvipdfmx, a4paper, report, papersize, 11pt]{jsbook}
\usepackage{bm}
\usepackage{amsmath}
\usepackage[dvipdfmx]{graphicx}
\usepackage{wrapfig}
\usepackage[hang, small, bf]{caption}
\usepackage[subrefformat=parens]{subcaption}
\usepackage{here}
\usepackage{comment}
\captionsetup{compatibility=false}

\bibliographystyle{junsrt}
\title{二色の光周波数コムによるレーザー冷却法の開拓}
\author{物理工学科4年 中西亮}
\date{2018/11/19}
\begin{document}
\maketitle
\newpage

\setcounter{tocdepth}{2}
\tableofcontents


\newpage
\chapter{光周波数コムのテーパーアンプによる増幅実験}
\section{光周波数コムのテーパーアンプによる増幅}
 二光子励起の励起効率の式(\ref{ResonanceRabi}), (\ref{ExcitationRate})から分かる通り、励起効率はコムの強度の二乗に比例する。そのため、二光子のレーザー冷却を行うに当たり、高効率の二光子励起を実現するためには高強度のレーザーを用意することが非常に重要である。そのため今回の実験では、光周波数コムから得られた光をTAを用いて増幅するという手段を用いる。しかし、通常TAはcwレーザーを増幅するために用いられるため、光周波数コムの増幅に用いた場合にどのような振る舞いを見せるかについての過去の研究は限られており、異なる繰り返し周波数に対してのTAの利得を調べた研究はまだない。今回の実験では繰り返し周波数の異なる光周波数コムに対してTAの増幅の様子を測定した。

\newpage
\section{テーパーアンプのマウンターの組み立て}
 今回の実験では、Cs原子のレーザー冷却に必要なパワーを得るためにTAを用いた。光周波数コムのから$760 \mathrm{nm}$付近の波長と$890 \mathrm{nm}$付近の波長を切り出し増幅した。$890 \mathrm{nm}$側の増幅に用いるTAのチップのマウンターに関しては、設計と組み立てを行った。TAのチップはeagleyard社のEYP-TPA-0915-01500-3006-CMT03-0000を用いた。TAのチップの構造は図\ref{TA_chip_ds}のようになっている。\\
\begin{figure}[htbp]
 \begin{center}
  \includegraphics[width=70mm]{figures/chapter4/TA_chip_ds.png}
\end{center}
 \caption{TAチップの構造図(eagleyard社のデータシートから引用)}
 \label{TA_chip_ds}
\end{figure}
 TAチップのマウンターの構造は図\ref{TA_mounter_photo_comments},\ref{TA_mounter_structure}のようになっている。ただし、TAチップの入力光をマスター光、出力光をスレーブ光と呼ぶ。TAチップは銅のブロックにコリメーションレンズ2枚と共に取り付けられており、そこに温度センサーと直流電源からのSMAケーブルが繋がっている。この銅製のブロックをアルミニウム製のブロックを介して光学定盤に固定している。二つのブロックの間にペルチェ素子を挟み、温度を制御している。なお、レンズのマウンターにはアルミニウムを使用している。

\begin{figure}[htpb]
  \centering
    \begin{tabular}{c}

%----- TAチップマウンターの概観 -----

      \begin{minipage}{0.50\hsize}
        \centering
          \includegraphics[keepaspectratio,  scale=0.30,  angle=0]
                          {figures/chapter4/TA_mounter_photo_comments.png}
                          \caption{TAチップマウンターの概観}
                          \label{TA_mounter_photo_comments}
      \end{minipage}

%----- TAチップマウンターの構造図 -----

      \begin{minipage}{0.50\hsize}
        \centering
          \includegraphics[keepaspectratio,  scale=0.35,  angle=0]
                          {figures/chapter4/TA_mounter_structure.png}
                          \caption{TAチップマウンターの構造図}
                          \label{TA_mounter_structure}
      \end{minipage}

    \end{tabular}
\end{figure}
\newpage
 なお、実際に使用する際には、図\ref{TA_case}のようにアクリルボードでケースを作り使用した。 また、スレーブ光の形状は光の回折の効果から楕円状になっているため、垂直方向のコリメーションを銅ブロック状のレンズで行い、水平方向のコリメーションを追加のシリンドリカルレンズで行う必要がある。\\
\begin{figure}[htbp]
 \begin{center}
  \includegraphics[width=75mm]{figures/chapter4/TA_case.jpg}
\end{center}
 \caption{実際に使用時のTAの様子}
 \label{TA_case}
\end{figure}



\newpage
\section{増幅実験の測定手法}
\subsection{繰り返し周波数の異なるコムでの増幅の比較}
 繰り返し周波数が$120$ MHzと$1.6$ GHzの二台のコムの、$766$ nmを中心波長とする幅$10$ nmのバンドパスフィルター(BPF)を通過した光をTAで増幅させ、マスター光とスレーブ光のパワーを測定した。その際に、BPF通過前のコムのスペクトルと、テーパーアンプの入り口と出口でのコムのスペクトルも測定を行った。その際の光学系は図\ref{760_amp_diagram}, \ref{760_astro_amp_diagram}に示している。繰り返し周波数$120$ MHzのコムを用いた実験ではアイソレータを入れないでTAの実験を行ったところ、TAからの戻り光がコムに光フィードバックをもたらしcw的発振を引き起こすことが観測された。このためアイソレータを使用している。一方で、くり返し周波数が$1.6$ GHzのコムの実験ではアイソレータを使用しなくても、TAからの戻り光がコムの共振器まで戻らずcw的な発振を起こさなかったため、アイソレータは使用しなかった。\\
\subsection{自作のTAでの890nm付近のコムの増幅}
 また、自作したTAで繰り返し周波数が$1.6$ GHzのコムの、$890$ nmを中心波長とする幅$10$ nmのBPFを通過した光を増幅させ、マスター光とスレーブ光のパワーの測定を行った。その際の光学系は図\ref{890_astro_amp_diagram}に示した。
\begin{figure}[htpb]
  \centering
    \begin{tabular}{c}
\begin{comment}
%----- 繰り返し周波数$120$ MHzのコムの$766$ nm付近の光を増幅した際の光学系 -----

      \begin{minipage}{0.5\hsize}
        \centering
          \includegraphics[keepaspectratio,  scale=0.16,  angle=0]
                          {figures/760_amp_experiment_comment.png}
                          \caption{繰り返し周波数$120$ MHzのコムの$766$ nm付近の光を増幅した際の光学系}
                          \label{760_amp_experiment_comment}
      \end{minipage}

%----- 上の写真の概略図 -----
\end{comment}

      \begin{minipage}{0.5\hsize}
        \centering
          \includegraphics[keepaspectratio,  scale=0.25,  angle=0]
                          {figures/chapter4/760_amp_diagram.png}
                          \caption{繰り返し周波数$120$ MHzのコムの$766$ nm付近の光を増幅した際の光学系\\(アイソレータ有り)}
                          \label{760_amp_diagram}
      \end{minipage}

      \begin{minipage}{0.5\hsize}
        \centering
          \includegraphics[keepaspectratio,  scale=0.33,  angle=0]
                          {figures/chapter4/760_astro_amp_diagram.png}
                          \caption{繰り返し周波数$1.6$ GHzのコムの$766$ nm付近の光を増幅した際の光学系}
                          \label{760_astro_amp_diagram}
      \end{minipage}\\
      \\

      \begin{minipage}{0.5\hsize}
        \centering
          \includegraphics[keepaspectratio,  scale=0.3,  angle=0]
                          {figures/chapter4/890_astro_amp_diagram.png}
                          \caption{繰り返し周波数$1.6$ GHzのコムの$890$ nm付近の光を増幅した際の光学系}
                          \label{890_astro_amp_diagram}
      \end{minipage}

    \end{tabular}
\end{figure}
\newpage
\subsection{ダブルパスによる増幅}
  TAは通常ゲイン領域が細い方からマスター光を入射し、ゲイン領域の広い方からスレーブ光を出力するという使用法をする。しかし、通常の出力側からマスター光を入射し通常の入力側から出た光をミラーで打ち返し再度通常の入力側に光を入射させ、二度ゲイン領域を通過させ増幅するという手法がある。この光学系の配置をダブルパスといい、これに対して通常の配置をシングルパスと呼ぶことがある。ダブルパスでのcwレーザーのTAの増幅の振る舞いについては過去の研究\cite{doi:10.1063/1.3501966}があり、通常cw光でTAを飽和させるには数十ワットのマスター光が必要となるが、ダブルパスの場合だと$200 \mathrm{\mu W}$のマスター光で飽和させることが出来ることが分かっている。また、スペクトルの面でもキャリアの周波数的に幅の広い自然放出が抑えられることが分かっている。このように、ダブルパスによるメリットは多いが光周波数コムの増幅にダブルパスを用いている研究はまだない。そのため今回の実験では,繰り返し周波数$1.6$ GHzの$761$ nmから$771$ nmのコムをダブルパスにより増幅する実験を行った。

 \begin{figure}[htbp]
  \begin{center}
   \includegraphics[width=120mm]{figures/chapter4/doublepass_photo.png}
 \end{center}
  \caption{ダブルパス配置の光学系}
  \label{doublepass_photo}
 \end{figure}

\newpage
\section{測定結果}
\subsection{TAの戻り光による光周波数コムのスペクトルの変化}
 図\ref{spectrum_current_MODORI}はマスター光入射時にアイソレータを使用しなかった時の、光周波数コムのキャビティの出力口におけるスペクトラムをTAに印加する電流の大きさを変えつつ分光器で測定したものである。TAの印加電流をあげていくと$770$ nm付近でcw的な発振を起こしていることが分かる。これはTAの入射口から出た自然放出の光が光周波数コムの共振器まで戻り、光フィードバックを起こしているものと考えられる。\\
 そのためマスター光入射時にアイソレータを通過させたところ、スペクトルは図\ref{spectrum_current_isolator}のようになった。このようなcw的な発振は観測されなかった。
\begin{figure}[H]
 \begin{center}
  \includegraphics[width=140mm]{figures/chapter4/spectrum_current_MODORI.png}
\end{center}
 \caption{戻り光の影響によるコムのスペクトルのTAの印加電流依存性}
 \label{spectrum_current_MODORI}
\end{figure}
\begin{figure}[H]
 \begin{center}
  \includegraphics[width=100mm]{figures/chapter4/comb-spectrum_no-return.png}
\end{center}
 \caption{アイソレータ使用時のコムのスペクトル。TAの印加電流は$1617$ mAとした。}
 \label{spectrum_current_isolator}
\end{figure}



\section{繰り返し周波数$120$ MHzの光周波数コムによる増幅実験}

\section{考察}
\subsection{注入される電子数とレーザーの光子数による考察}

$f_{\mathrm{rep}} = 120$ MHzのコムでは利得の飽和が観測されたが、二光子冷却の励起効率を考えたときに必要なスレーブ光のパワーが得られていない。そこで、TAによるコムの増幅において利得を上げるためにはどのような手法が必要となるのかを簡単なモデルを用いて原理的に考察を行う。\par
図のような直方体のゲイン領域をもつ半導体の増幅素子を考える。この素子には電流$I$によってキャリアとホールが注入されており、これらの再結合によって光子を生成するメカニズムとなっている。このキャリアの再結合のプロセスにはマスター光による誘導放出に加えて自然放出や自然放出の光子による誘導放出など複数のものがあるが、これらを全てまとめた実効的なキャリアの寿命を$\tau$とする。このとき、キャリア密度$n_{\mathrm{c}}$の時間変化を表す微分方程式は
\begin{equation}
  \frac{dn_{\mathrm{c}}}{dt} = \eta \frac{I}{eV}-\frac{n_{\mathrm{c}}}{\tau}
\end{equation}
と書ける。ここで、$\eta$は量子効率、$e_0$は電気素量、$V$はゲイン領域の体積を表す。この微分方程式を初期条件$n_c = 0$の下で解くと
\begin{equation}
  n_c = \frac{\eta\tau I}{e_0V}\left(1-e^{-\frac{t}{\tau}}\right)
\end{equation}
キャリア密度の上限は印加電流の大きさとキャリアの寿命にによって決まることが分かる。キャリア密度の時間変化は図\ref{carrier_saturation}のようになるので、キャリア密度の飽和はキャリアの寿命程度の時間でおこることが分かる。\\
\begin{figure}[H]
 \begin{center}
  \includegraphics[width=100mm]{figures/chapter4/carrier_saturation.png}
\end{center}
 \caption{キャリア密度の時間変化}
 \label{carrier_saturation}
\end{figure}
cw光をTAで増幅した場合、通常$10$ mW程度のマスター光を$1$ A程度の印加電流で$1$ W程度まで増幅することができる。このとき、$766$ nmのスレーブ光に単位時間あたり含まれる光子数を計算すると、
\begin{equation}\label{TA_photon_rate}
  \frac{1\ \mathrm{W}}{1.6\ \mathrm{e_0V}\times1.6\times10^{-19}\ \mathrm{C}} = 4\times10^{18}\ 個\mathrm{/s}
\end{equation}
となる。これに対して$1$ Aの電流から毎秒含まれる電子数を計算すると、
\begin{equation}
  \frac{I}{e_0} = \frac{1\ \mathrm{A}}{1.6\times10^{-19}\ \mathrm{C}} = 6\times10^{18}\ 個\mathrm{/s}
\end{equation}
となる。両者を比較するとcw光のTAによる増幅においては注入された電子が効率よく光子に変換できていることが分かる。\par
これに対して、コムのパルスの増幅を行う場合を考える。キャリアの寿命を$500$ psと仮定するとこの時間でTAが供給できるフォトン数は式(\ref{TA_photon_rate})を用いると、
\begin{equation}
  4\times10^{18}\times500\times10^{-12} = 2\times10^9 個
\end{equation}
と計算できる。
 これに対して、$f_{\mathrm{rep}} = 120$ MHzのコムが$120$ mWの平均パワーを持つ時の一つのパルスのエネルギーは$10$ pJとなる。一つのパルスに含まれる光子の数は
\begin{equation}
  \frac{10\ \mathrm{pJ}}{1.5\ \mathrm{eV}\times1.6\times10^{-19}} = 4\times10^7 個
\end{equation}
となる。よってこの計算では増幅の利得は$50$倍程度となる。実際にはASEなどの効果でキャリアの寿命はより短くなると考えられるので、パルスに含まれる光子の数に対して平衡状態にあるキャリアの数が不足していることが前節での実験で利得の上限を定めていると考えられる。また、典型的な半導体のキャリアの寿命は数百psに対して$f_{\mathrm{rep}} = 120$ MHzの場合の繰り返し時間は$T_\mathrm{r} = 8$ ns程度のため、パルス通過後から次のパルスが到達するまでの間に注入されたキャリアの再結合が進んでしまっていると考えられる。これらの考察から、パルス当たりのエネルギーを低下させ、繰り返し周波数を高めることで利得の向上を見込める。
\subsection{TAのパルスの伝播の数式的なモデル}
レーザーパルスがゲイン領域の内部を群速度$v_\mathrm{g}$程度で伝搬しつつ増幅する様子を表す偏微分方程式は
\begin{equation}
  \left(\frac{\partial}{\partial z} + \frac{1}{v_{\mathrm{g}}}\frac{\partial}{\partial t}\right)P(z,t) = g(z,t)P(z,t)
\end{equation}
と書ける。これに加えて、キャリア密度については
\begin{eqnarray}
  \frac{dn}{dt} &=& \eta \frac{I}{eV} - \frac{n}{\tau(n)}-v_{\mathrm{g}}\sigma(n(z,t))n_{\mathrm{p}}\\
  n_{\mathrm{p}} &=& \frac{P}{\hbar\omega v_{\mathrm{g}}A}
\end{eqnarray}
という方程式を立てることができる。ただし、$n$はキャリア密度、$n_{\mathrm{p}}$は光子密度、$\sigma$は共鳴散乱の効率、$A$はゲイン領域の断面積を表す。利得$g(n(z,t))$については、 
\begin{equation}
  \sigma(n(z,t)) = \sigma_0 \ln{\left( \frac{n(z,t)}{n_0}\right)}
\end{equation}
で与えられる。ここでパラメータ$\sigma_0$は物質固有の定数である。$n_0$は吸収に寄与するキャリアの数である。
\begin{comment}
$\sigma_0$は$1000-5000\ \mathrm{cm^{2}}$程度の値を取ることがわかっている.
\end{comment}
\begin{comment}
$n_0$は$1.5-3\times10^{18}\ \mathrm{cm^{-3}}$程度の値となることが知られている。
\end{comment}
ただし、この微分方程式を差分法で解くと解が発散してしまうため、数値的に解くことは今後の課題として残っている。
\subsection{繰り返し周波数の異なるコムの増幅}

図\ref{pulse_power-gain-comparison}は766nm側のコムを増幅した際の利得のパルスエネルギー依存性依存性を二つの繰り返し周波数に応じて比較したものである。繰り返し周波数$120 \mathrm{MHz}$の利得をみると、パルスエネルギーの増加に対して利得が低下していく様子がわかる。これは、各パルスに含まれる光子数に対して励起状態にあるキャリア数が足りておらずパルスエネルギーの増加に対して利得を保てていないと考えられる。それに対し、繰り返し周波数$1.6 \mathrm{GHz}$のコムの利得は$120$ MHzのコムの利得を下回っている。これはパルスの時間間隔が$630$ ps程度で短く、十分な反転分布が励起されていないことが原因ではないかと考えられる。\\
 また、スレーブ光強度のマスター光強度依存性を二台のコムで比較すると図\ref{M-S_power-comparison}のようになる。同じマスター光強度で比較すると、繰り返し周波数が$1.6$ GHzのコムのスレーブ光強度が上回っていることがわかる。これは同じ光強度の場合、繰り返し周波数が$120$ MHzのコムでは繰り返し周波数が$1.6$ GHzのコムに比べ一つのパルスに含まれる光子数が多いが、TA内のキャリア数が少ないので誘導放出に使われない光子数が多くなってしまい光が増幅されないことが原因だと考えられる。一方、$1.6$ GHzのコムでは一パルスあたりの光子数が少ないため、無駄になる光子が少なく効率よく増幅することができると考えられる。

\begin{figure}[htpb]
  \centering
    \begin{tabular}{c}

%----- 写真 -----

      \begin{minipage}{0.50\hsize}
        \centering
          \includegraphics[keepaspectratio,  scale=0.30,  angle=0]
                          {figures/chapter4/pulse_power-gain-comparison.png}
                          \caption{二台のコムにおけるTAの利得のパルスエネルギー依存性の比較}
                          \label{pulse_power-gain-comparison}
      \end{minipage}

%----- PD Signal -----

      \begin{minipage}{0.50\hsize}
        \centering
          \includegraphics[keepaspectratio,  scale=0.35,  angle=0]
                          {figures/chapter4/M-S_power_comparison.png}
                          \caption{二台のコムにおけるスレーブ光強度のマスター光強度依存性の比較}
                          \label{M-S_power-comparison}
      \end{minipage} \\

      \begin{minipage}{0.50\hsize}
        \centering
          \includegraphics[keepaspectratio,  scale=0.35,  angle=0]
                          {figures/chapter4/TA_cuurent-gain_comparison.png}
                          \caption{二台のコムにおけるTAの利得の印加電流依存性の比較}
                          \label{TA_cuurent-gain_comparison}
      \end{minipage}
    \end{tabular}
\end{figure}
\subsection{繰り返し周波数$1.6$ GHzの$890$ nm付近のコムの増幅}
 繰り返し周波数$1.6$ GHzのコムの$885 \mathrm{nm}から 895 \mathrm{nm}$のBPFを通過させた光をTAで増幅した結果を、図\ref{TA_power-current_3A_astro}, \ref{890TPA_power_dependence_0117}に示す。

\begin{figure}[H]
  \centering
    \begin{tabular}{c}

%----- 写真 -----

      \begin{minipage}{0.50\hsize}
        \centering
          \includegraphics[keepaspectratio,  scale=0.50,  angle=0]
                          {figures/chapter4/TA_power-current_3A_astro.png}
                          \caption{890nm側の繰り返し周波数$1.6$ GHzでのTAのスレーブ光強度の電流依存性}
                          \label{TA_power-current_3A_astro}
      \end{minipage}

%----- PD Signal -----

      \begin{minipage}{0.50\hsize}
        \centering
          \includegraphics[keepaspectratio,  scale=0.5,  angle=0]
                          {figures/chapter4/890TPA_power_dependence_0117.png}
                          \caption{890nm側の繰り返し周波数$1.6$ GHzでのTAのスレーブ光強度のマスター光強度依存性}
                          \label{890TPA_power_dependence_0117}
      \end{minipage} \\

    \end{tabular}
\end{figure}

\subsection{ダブルパスでの増幅}
 ダブルパスでの増幅の様子を図\ref{double-pass_I-Gain},\ref{double-pass_I-Slave}に示す。図\ref{double-pass_I-Slave}で示されたスレーブ光のスペクトルをみると印加電流が$930$ mA以上ではスレーブ光がcw的に発振してしまっていることが分かる。これは、マスター光が$116 \mathrm{\mu W}$と小さなパワーしか用意することができなかったため、使用したTAのゲイン特性の偏りからゲインの高い波長で優先的に自然放出が起きてしまったものと考えられる。

\newpage
\begin{figure}[htpb]
  \centering
    \begin{tabular}{c}
      \begin{minipage}{0.50\hsize}
        \centering
          \includegraphics[keepaspectratio,  scale=0.5,  angle=0]
                          {figures/chapter4/double-pass_I-slavepower.png}
                          \caption{ダブルパスでのスレーブ光強度の印加電流依存性}
                          \label{double-pass_I-slavepower}
      \end{minipage}
      \begin{minipage}{0.50\hsize}
        \centering
          \includegraphics[keepaspectratio,  scale=0.5,  angle=0]
                          {figures/chapter4/double-pass_I-Gain.png}
                          \caption{ダブルパスでの利得の印加電流依存性}
                          \label{double-pass_I-Gain}
      \end{minipage}\\

      \begin{minipage}{1\hsize}
        \centering
          \includegraphics[keepaspectratio,  scale=0.340,  angle=0]
                          {figures/chapter4/double-pass-Slave-Spectrum.png}
                          \caption{ダブルパスでの各印加電流におけるスレーブ光のスペクトル}
                          \label{double-pass_I-Slave}

      \end{minipage}


    \end{tabular}
\end{figure}


\bibliography{reference}
\end{document}
