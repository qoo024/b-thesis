\documentclass[uplatex, dvipdfmx, a4paper, report, papersize, 11pt]{jsbook}
\usepackage{bm}
\usepackage{amsmath}
\usepackage[dvipdfmx]{graphicx}
\usepackage{wrapfig}
\usepackage[hang, small, bf]{caption}
\usepackage[subrefformat=parens]{subcaption}
\usepackage{here}
\usepackage{comment}
\captionsetup{compatibility=false}

\bibliographystyle{junsrt}

\title{\fontsize{24.88pt}{0pt}\selectfont 卒業論文 \vspace{2cm}\\ 二色の光周波数コムによる\vspace{4mm}\\ レーザー冷却法の開拓 \vspace{3cm}\\ \fontsize{17.28pt}{0pt}\selectfont 指導教員\ \ \ \ \ 吉岡孝高\ \ 准教授 \vspace{4cm}\\ \fontsize{17.28pt}{0pt}\selectfont  平成31年2月提出 \vspace{2cm} \\ \fontsize{17.28pt}{0pt}\selectfont 東京大学工学部物理工学科 \vspace{3mm} \\ 03-170579\ \ \ 中西亮}
\date{}

\begin{document}
\begin{center}
  \Huge 卒業論文 \par
  \vspace{20mm}
  \Huge 二色の光周波数コムによる \par
  \vspace{4mm}
  \Huge レーザー冷却法の開拓 \par
  \vspace{30mm}
  \LARGE 指導教員\ \ \ \ \ 吉岡孝高\ \ 准教授\par
  \vspace{30mm}
  \LARGE 平成31年2月提出\par
  \vspace{15mm}
  \LARGE 東京大学工学部物理工学科 \par
  \vspace{10mm}
  \LARGE 03-170579\ \ \ 中西亮
  \vspace{10mm}
\end{center}
\thispagestyle{empty}
\clearpage
\addtocounter{page}{-2}
\newpage
\thispagestyle{empty}
 
\newpage


\setcounter{tocdepth}{2}
\tableofcontents


\newpage
\chapter{序論}
\section{研究の背景と目的}
\subsection{レーザー冷却の研究の現在と課題}
レーザー光の輻射圧を利用して原子の運動を抑制する技術であるレーザー冷却は原子物理において有用な手法として大きな役割を果たしている. 1980年以降, 研究が本格化したレーザー冷却は現在, ボースアインシュタイン凝縮の実現や光格子時計の精度向上, 量子情報処理への利用など様々な応用が研究されている\cite{レーザー冷却とその応用}. このようにレーザー冷却は, レーザーの特性を活用して原子の運動や内部状態の高度な操作を可能にした革新的技術であり, 現在では冷却原子分野の基礎技術としてすでに定着しているものである. \par
しかし, 連続波発振(cw)レーザーを用いるこれまでの冷却法によって, あらゆる原子を冷却できるわけではない. 現在のレーザーで冷却することができる原子の種類は非常に限られており, 主にアルカリ金属, アルカリ土類金属の原子と希ガス原子の準安定状態である \cite{PhysRevA.73.063407}. そして, 炭素や酸素など化学や生物の分野で重要な原子の冷却はまだできていない. この原因として, 多くの原子の冷却に必要とされる深紫外領域での高出力のcwレーザーを現状では実現できていないこと, 価電子の多い原子においては閉じた電子の遷移サイクルを実現するために多数のリポンプレーザーが必要となり光学系が複雑化してしまうことが挙げられる.\par
また, 類似の問題は分子のレーザー冷却にも見られる. 近年分子のレーザー冷却や磁気光学トラップの構築も精力的に進められているが, 分子特有の多彩な振動回転順位の存在に起因して, 様々な光周波数のcwレーザーを用意しなければ効果的な冷却ができないことが知られており, 実際に実験系は複雑なものとなっている.


\subsection{今回の研究の目的}
特に, 現存のcwレーザーでは達成することのできない短波長が必要なレーザー冷却を実現しうる方法として, 高いピークパワーをもつ光周波数コムによる二光子遷移を利用したレーザー冷却の手法が提案されており, Rbを用いた原理検証実験も行われている\cite{PhysRevA.73.063407, PhysRevX.6.041004}. しかし, 一種類のモード同期レーザーを用いた過去の実証実験の方法では, 電子準位を近共鳴中間状態として用いることのできる原子は限られてしまう. そこで本論文では, より汎用性が高く, 多彩な原子系に光周波数コムを用いた二光子冷却を適用可能にするために, 二色の光周波数コムを用いる構成に着目して, Cs原子を対象にその妥当性を検証し, 実際に二色の光周波数コムを用いてレーザー冷却を実証することを目的とした. \\

\section{本論文の構成}
 まず, 第2章で今回の研究で用いられている手法の基礎的な事項について紹介を行っている. 第3章では先行研究による光周波数コムを用いた二光子遷移の理論を紹介し, Cs原子のコムによる二光子冷却に必要な励起効率を得るために, どのような光周波数コムを準備すれば良いのかについてシミュレーションを用いて検討を行う. 第4章ではCs原子のコムによる冷却の前に行う, 予冷のための磁気光学トラップに用いるcw光源の構築の原理と実験系について紹介し, 実際にロックした際の線幅を評価する. 第5章では, 二色のコムによる冷却のための光源に必要なパワーを得るためにテーパーアンプを用いて行った増幅実験の実験結果とその考察についてまとめている. 最後に第6章で, 本論文のまとめと今後の展望について述べる.
\newpage

\chapter{背景知識}

\section{ドップラー冷却}
\subsection{光が原子に及ぼす力}
 光が原子に及ぼす力は輻射圧と双極子力に分けることができる. 輻射圧は, 光子を吸収・放出する際に原子の運動量変化を起こす撃力である\cite{ノーベル賞と分光学}. 双極子力は, 光電場により誘起された原子の双極子モーメントが光電場強度の空間的な不均一性を感じることに起因する保存力である\cite{気体原子のレーザー冷却}.
 電磁場の周波数を$\omega$, 原子の共鳴周波数を$\omega_0$としたとき, 波数$\bm k$の平面電磁場が速度$\bm v$の原子に与える平均輻射圧は,
 \begin{equation}\label{scattering_force}
\bm{F} _ { \mathrm { scatt } } = \hbar \bm{k}\frac { \Gamma } { 2 } \frac { \Omega ^ { 2 } / 2 } { \delta ^ { 2 } + \Omega ^ { 2 } / 2 + \Gamma ^ { 2 } / 4 }
 \end{equation}
となる\cite{Foot:1080846}. ただし, $\Gamma$は遷移の自然幅(半値全幅), $\Omega$は共鳴ラビ周波数, $\delta = \omega - \omega _ { 0 } - \bm{k} \cdot \bm{v}$はドップラー効果を考慮に入れた離調である. もう一つの力である双極子力は, 保存力であるために原子の捕獲はできるが冷却はできない\cite{ノーベル賞と分光学}.

\subsection{光糖蜜効果}
ドップラー冷却は原子の共鳴周波数に対して負の離調をつけた直交する3組の対向レーザーを, 原子に照射することで実現する. ある速度をもつ原子は, ドップラー効果により原子の運動と反対向きのレーザーの周波数をより高く感じ, 原子の運動と同じ向きのレーザーの周波数をより低く感じる. このため共鳴周波数に負の離調をつけた対向レーザーを原子に対して当てると, 原子は運動と反対向きのレーザーの輻射圧をより強く感じるので, 原子の速度は減速される. このように, 原子がレーザーを吸収するときに受ける力は常に運動の向きと逆向きであるが, 励起された原子が光子を自然放出する向きはランダムであるため自然放出による運動量の変化は平均すると無くなる. このため, 原子の運動は光子の吸収と放出を繰り返すことで抑えられていくことが分かる.\\
 輻射圧の式(\ref{scattering_force})から, 具体的に1組の対向ビームが原子に対して与える力を計算すると,
\begin{equation}
  \begin{split}
    F _ { \mathrm{molasses} } &= F _ { \mathrm{scatt}  } \left( \omega - \omega _ { 0 } - k v \right) - F _ {  \mathrm{scatt} }  \left( \omega - \omega _ { 0 } + k v \right)
    \\& \simeq F _ { \mathrm{scatt}  } \left( \omega - \omega _ { 0 } \right) - k v \frac { \partial F } { \partial \omega } - \left[ F _ {  \mathrm{scatt} } \left( \omega - \omega _ { 0 } \right) + k v \frac { \partial F } { \partial \omega } \right]
    \\& \simeq - 2 \frac { \partial F } { \partial \omega } k v
  \end{split}
\end{equation}
となる. この式を見ると, 対向ビームが原子に与える力は速度に比例する粘性抵抗の形をしていることが分かる.3組の対向ビームを当てると原子は粘性液体の中を動いているような振る舞いを見せる. このことからこの効果を初めて実証したChuらは, 「光糖蜜(Optical Molasses,  OM)効果」と呼んだ.

\subsection{ドップラー冷却限界}
 励起された原子は光子を自然放出する際に反跳を受ける.この反跳の向きが毎回ランダムなので原子はこの自然放出により加熱効果を受ける. また単位時間あたりに吸収する光子の数にも揺らぎがあるため, 原子にランダムな運動を引き起こす.
ドップラー冷却にはこのような加熱効果が存在し, 前述の冷却効果と均衡が取れるところがドップラー冷却で到達できる温度の限界となる. この温度$T _ { \mathrm { D } }$は
\begin{equation}
  k _ { \mathrm { B } } T _ { \mathrm { D } } = \frac { \hbar \Gamma } { 2 }
\end{equation}
で与えられる. ただし, $k _ { \mathrm { B } }$はボルツマン定数を表す. 典型的なドップラー冷却限界温度の例としては, Naは$240\ \mathrm{\mu K}$, Rbは$146\ \mathrm{\mu K}$, Liは約$140\ \mathrm{\mu K} $である\cite{PhysRevX.6.041004, 気体原子のレーザー冷却, ubcthesis}.
\subsection{ドップラー冷却できる原子}
 ドップラー冷却は原理的にはどの原子に対しても使うことができるが, 実験上の制約から以下のような特徴をもつ原子に適用が制限される\cite{ノーベル賞と分光学}.
\begin{itemize}
  \item 光の吸収放出のサイクルが速い
  \item 冷却サイクルが閉じている.あるいは少数のレーザーで閉じることができる.
  \item 目的とする周波数帯に使いやすいレーザーが存在する.
\end{itemize}

\newpage


\section{磁気光学トラップ (MOT)}
 光糖蜜効果によるドップラー冷却では原子を減速させ冷却することはできるが, トラップすることはできない. 磁気光学トラップ(Magneto-Optical Trap,  MOT)は原子をトラップするための技術の一つである. MOTは図\ref{MOT_circular_polarization}のように, OMの3組の負の離調をつけた円偏光対向レーザーと, 一対のコイルに逆向きの電流を流すことで導入される四重極磁場(図\ref{MOT_magnetic_field})で構成される. 四重極磁場による原子準位のゼーマン分裂を利用して原子に加わる輻射圧に位置依存性をつけることで原子をトラップする.\\
 簡単な例として, 基底状態が$J = 0$, 励起状態が$J = 1$の遷移を考える.トラップの中心では磁場は$0$になり, 中心付近では磁場の大きさは中心からの変位に比例する. そのため, 四重極磁場によるゼーマン分裂は図\ref{MOT_zeeman_split}のようになる. また, 選択則により$\sigma^{+}(\sigma^{-})$の偏光は$m_J = 0$から$m_J = 1 (-1)$に励起する. 原子が$z > 0$の位置に移動すると, $m_J = -1$の準位のエネルギーが下がり, 負の離調をつけた$\sigma^-$の輻射圧を強く受けてトラップの中心に押し戻される. $z < 0$に移動した時も同様の原理で中心に押し戻す力が働くことで原子がトラップされる.\\
  式(\ref{scattering_force})を用いて, 原子に及ぼされる輻射圧を計算すると,
\begin{equation}
  \begin{aligned}
     F _ { \mathrm { MOT } } & = F _ { \mathrm { scatt } } ^ { \sigma ^ { + } } \left( \omega - k v - \left( \omega _ { 0 } + \beta z \right) \right) - F _ { \mathrm { scatt } } ^ { \sigma ^ { - } } \left( \omega + k v - \left( \omega _ { 0 } - \beta z \right) \right) \\ & \simeq - 2 \frac { \partial F } { \partial \omega } k v + 2 \frac { \partial F } { \partial \omega _ { 0 } } \beta z
  \end{aligned}
\end{equation}
となる. ここで$\beta z$の項はゼーマンシフトを表し,
\begin{equation}
  \beta z = \frac { g \mu _ { \mathrm { B } } } { \hbar } \frac { \mathrm { d } B } { \mathrm { d } z } z
\end{equation}
ただし, ここでは$g = g _ { J }$でランデの$g$因子を表し, $\mu _ { \mathrm { B }}$はボーア磁子である. いま, 力は$\delta = \omega - \omega _ { 0 }$に依存しているので, $\partial F / \partial \omega _ { 0 } = - \partial F / \partial \omega$より,
\begin{equation}
  \begin{aligned} F _ { \mathrm { MOT } } & = - 2 \frac { \partial F } { \partial \omega } ( k v + \beta z ) \\ & = - \alpha v - \frac { \alpha \beta } { k } z \end{aligned}
\end{equation}
となる. ただし$\alpha = 2\frac{\partial F}{\partial \omega} k$である. この表式から, MOT中の原子には復元力と粘性抵抗の形で表される力が働いていることが分かる.
\newpage

\begin{comment}
\begin{figure}[htbp]
 \begin{center}
  \includegraphics[width=40mm]{figures/MOT_circular_polarization.png}
\end{center}
 \caption{MOT}
 \label{MOT_circular_polarization}
\end{figure}

\begin{figure}[htbp]
 \begin{center}
  \includegraphics[width=50mm]{figures/MOT_magnetic_field.png}
\end{center}
 \caption{四重極磁場}
 \label{MOT_magnetic_field}
\end{figure}

\begin{figure}[htbp]
 \begin{center}
  \includegraphics[width=50mm]{figures/MOT_zeeman_split.png}
 \end{center}
 \caption{四重極磁場によるゼーマン分裂}
 \label{MOT_zeeman_split}
\end{figure}
\end{comment}


\begin{figure}[htpb]
  \centering
    \begin{tabular}{c}

%----- MOTの構成 -----

      \begin{minipage}{0.50\hsize}
        \centering
          \includegraphics[keepaspectratio,  scale=0.35,  angle=0]
                          {figures/chapter2/MOT_circular_polarization.png}
                          \caption{MOTの構成(文献\cite{Foot:1080846}より引用)}
                          \label{MOT_circular_polarization}
      \end{minipage}

%----- 四重極磁場 -----

      \begin{minipage}{0.50\hsize}
        \centering
          \includegraphics[keepaspectratio,  scale=0.50,  angle=0]
                          {figures/chapter2/MOT_magnetic_field.png}
                          \caption{四重極磁場(文献\cite{Foot:1080846}より引用)}
                          \label{MOT_magnetic_field}
      \end{minipage} \\

%-----四重極磁場によるゼーマン分裂 -----

      \begin{minipage}{0.50\hsize}
        \centering
          \includegraphics[keepaspectratio,  scale=0.40,  angle=0]
                          {figures/chapter2/MOT_zeeman_split.png}
                          \caption{四重極磁場によるゼーマン分裂(文献\cite{Foot:1080846}より引用)}
                          \label{MOT_zeeman_split}
      \end{minipage}


    \end{tabular}
\end{figure}

\newpage
\section{飽和吸収分光法}
\subsection{原理}
今回の実験の目標はコムでレーザー冷却をすることで, まずcw光によるMOTを構成しそこにコムの光を当てる. ここで, MOTを構成するために使うcw光の周波数をロックするために「飽和吸収分光法」を利用している. 通常の分光法では原子の遷移周波数の線幅は, 温度に応じたドップラー拡がりを持っている. これに対し, 飽和吸収分光法はドップラー拡がりを克服し, 原子の遷移周波数を測定する手法である. \par
図\ref{saturated_absorption_figure}のように, レーザーをポンプ光とプローブ光に分けて原子の気体が入ったセルに照射し, プローブ光のセルの透過強度を測定する.\par
レーザーの周波数が原子の共鳴周波数$\omega_0$に合っているとき, レーザーの進行方向に対して速度ゼロ近傍の原子が平均パワーの大きなポンプ光によって吸収飽和を生じているため, セル通過後のプローブ光はポンプ光の影響を受けてあまり原子に吸収されない. レーザーの周波数が$\omega_0$からずれているときは, 吸収飽和を生じているのは有限のドップラーシフトをもつ原子であるため, 逆方向から進行してくるプローブ光に対してはポンプ光の影響はない. この効果により, レーザーの周波数を掃引すると, プローブ光の透過光強度は図図\ref{PD_Signal_main}のようになり, ドップラーフリーの吸収スペクトルを得ることができる.
\subsection{クロスオーバー共鳴}
 飽和吸収分光においては, 準位間の共鳴周波数だけでなく二つの共鳴周波数のちょうど平均の周波数においてもプローブ光の透過率が極大となる. この現象をクロスオーバー共鳴と呼ぶ. もっとも簡単な三準位の系での飽和吸収分光を考える. 図\ref{crossover-diagram}のようなエネルギー準位において, $|1\rangle$から$|2\rangle$,$|3\rangle$への遷移周波数を$\omega_{12}$, $\omega_{13}$とする. レーザーの周波数が$\omega$のときに$|1\rangle$から$|2\rangle$へ励起される原子の速度は$v = \frac{\omega - \omega_{12}}{k}$であり, 同様に$v = \frac{\omega - \omega_{13}}{k}$の速度の原子が$|3\rangle$へ励起される.
$\omega = \frac{\omega_{12} + \omega_{13}}{2}$のとき, $|2\rangle$, $|3\rangle$へ励起される原子の速度が一致する. このため, $\omega_{12}$, $\omega_{13}$においてだけでなく, $\frac{\omega_{12} + \omega_{13}}{2}$においてもポンプ光による飽和の影響でプローブ光の透過率が極大となることが分かる.
\begin{figure}[H]
  \centering
    \begin{tabular}{c}

%----- 写真 -----

      \begin{minipage}{1\hsize}
        \centering
          \includegraphics[keepaspectratio,  scale=0.2,  angle=0]
                          {figures/chapter2/saturated_absorption_figure.png}
                          \caption{飽和吸収分光の光学系.
                          ただし, PBSはPolarization Beam Splitter, PDはPhoto Detectorを意味する. }
                          \label{saturated_absorption_figure}
      \end{minipage}\\

%----- PD Signal ----

      \begin{minipage}{1\hsize}
        \centering
          \includegraphics[keepaspectratio,  scale=0.5, angle=0]
                          {figures/chapter2/PD_Signal_main.png}
                          \caption{飽和吸収分光によって観測されたCsの超微細構造. 光源として用いているECDLの周波数を三角波で掃引しているため, ECDLの周波数が時間に比例して変化している. また, 縦軸の信号は値が小さいほど透過光のパワーが強くなっている. }
                          \label{PD_Signal_main}
      \end{minipage} \\
      \begin{minipage}{0.50\hsize}
        \centering
          \includegraphics[keepaspectratio,  scale=0.3,  angle=0]
                          {figures/chapter2/crossover-diagram.png}
                          \caption{3準位系の図}
                          \label{crossover-diagram}
      \end{minipage}
    \end{tabular}
\end{figure}

\newpage
\section{光周波数コム}
\subsection{モード同期レーザーの概要}
今回の実験で光源として用いられているモード同期レーザーは, 時間的に等間隔で出力されるパルス列である. このパルス列は共振器内の異なる縦モードの位相を同期することで得られる. モード同期の手法には能動的モード同期と受動的モード同期があるが, 今回の実験では受動的モード同期を用いている.\\
 受動的モード同期法は共振器内に可飽和吸収体を設置することで行われる. これは強度が弱い光は吸収するが, 高強度の光に対しては飽和を起こし高い透過率を示すような素子である. これによりパルスの裾では吸収が起こりパルスの頂点付近では吸収が起きないため, パルスの持続時間を短く保つことができる. このようにして, モード同期が自発的に行われる.\\
 レーザーの利得媒質としてチタンサファイア(Ti:Sa)を用いると, Ti:Sa自体が可飽和吸収体として振る舞う.これはTi:Sa中で起こるカーレンズ効果により, パルスの頂点付近の高強度の光は細くなり励起光との重なりが大きくなる分利得が増大するが, 裾野の光は重なりが小さく利得も比較的下がることでTi:Saが実効的な可飽和吸収体として振る舞うためである.\\
 また, 共振器内のミラーや利得媒質による群速度分散をプリズムやミラーで補償することで短パルスを維持する手法も用いられている. このように, フェムト秒パルス光の方がレーザー共振器内の損失が小さくなる効果と, 分散媒質をフェムト秒パルス光が通過する際に不可避であるチャープなどの分散の影響を補償することによって, フェムト秒パルス光が安定に発振するレーザーを実現できる.
\subsection{モード同期レーザーのスペクトル}
 モード同期レーザーの周波数領域でのスペクトルは等間隔のピークを持つ櫛状の形状となり, その周波数は
\begin{equation}
  f = nf_{\mathrm{rep}} + f_{\mathrm{ceo}}
\end{equation}
と表される\cite{Femtosecondopticalfrequencycombs}. $n$は整数である. ここで$f_{\mathrm{rep}}$は繰り返し周波数, $f_{\mathrm{ceo}}$はキャリア・エンベロープ・オフセット周波数と呼ばれる.このようなスペクトルの形状からモード同期レーザーは光周波数コムとも呼ばれる.\\
 また, $f_{\mathrm{ceo}}$はパルス間の位相のシフトに対応しており,
 \begin{equation}
   f_{\mathrm{ceo}} = \frac { 1 } { 2 \pi } f _ { \mathrm { rep } } \Delta \phi _ { \mathrm { ce } }
 \end{equation}
と表される\cite{Femtosecondopticalfrequencycombs}. ここで, $\Delta \phi _ { \mathrm { ce } }$はパルス間でのキャリアと包絡関数の位相差の変化を表す. $\Delta \phi _ { \mathrm { ce } }$は, 共振器内の分散による群速度と位相速度の差から生まれる. パルスは共振器内を一周する度に出力されるため, 位相の変化は
\begin{equation}
  \Delta \phi _ { \mathrm { ce } } = \left( \frac { 1 } { \nu _ \mathrm{ g } } - \frac { 1 } { \nu _ \mathrm{ p } } \right) l _ \mathrm{ c } \omega _ \mathrm{ c }\ (\bmod 2 \pi)
\end{equation}
と表される\cite{Femtosecondopticalfrequencycombs}. ただし, $\nu _ \mathrm{ g }$, $\nu _ \mathrm{ p }$はそれぞれ群速度, 位相速度を表し, $l_c$は共振器長, $\omega_c$はキャリア周波数を表す.

\begin{comment}
\section{テーパーアンプ}
今回の実験では光周波数コムの光をテーパーアンプ(Tapered Amplifier,  TA)で増幅しているが, この節ではTAの原理について説明する.

\subsection{半導体レーザーの発光原理}
TAは半導体で構成されており, その発光原理は半導体レーザーと同様であり構造も非常に似通っている. このため, まず文献\cite{わかる半導体レーザーの基礎と応用}の議論に沿って, 半導体レーザーの発光原理について説明する. 半導体レーザーの基本構造は図\ref{semicon_structure}のように, 3層のサンドイッチ構造で出来ており, 真ん中に挟まれ発光する層を「活性層」, 活性層を挟み込んで光やキャリアを閉じ込める役割を担う二つの層を「クラッド層」と呼ぶ. このような構造をDH(Double Hetero)構造と呼ぶ.\\
 この半導体p-n接合のエネルギーバンド構造は図\ref{semicon_bands}のようになる. 半導体レーザーでは図\ref{semicon_bands}のように電子が伝導帯から価電子帯に落ちる際のエネルギー差に相当する光が放出されるが, バンド間遷移であるためにライン間遷移に比べ発光スペクトルは広い. この素子に図\ref{semicon_structure}のように順方向のバイアスをかけるとエネルギーバンドが平らになっていき, p-n接合領域に電子とホールが集まり反転分布が形成される. このバンドギャップはGaAsの場合, 約$1.4$ eVで$800$ nmから$900$ nmに相当する\cite{grynberg_aspect_fabre_cohen-tannoudji_2010}. 半導体レーザーでは活性層の両へき界面に鏡をコーティングすることで光フィードバックを実現しレーザー動作を可能としている.

\begin{figure}[htpb]
  \centering
    \begin{tabular}{c}

%----- 半導体レーザーの構造 -----

      \begin{minipage}{0.70\hsize}
        \centering
          \includegraphics[keepaspectratio,  scale=0.35,  angle=0]
                          {figures/chapter2/semicon_structure.png}
                          \caption{半導体レーザーの基本構造}
                          \label{semicon_structure}
      \end{minipage}\\
      \\
      \\

%----- バンド図 -----

      \begin{minipage}{0.70\hsize}
        \centering
          \includegraphics[keepaspectratio,  scale=0.30,  angle=0]
                          {figures/chapter2/semicon_bands.png}
                          \caption{半導体レーザーのバンド図}
                          \label{semicon_bands}
      \end{minipage}

  \end{tabular}
\end{figure}
\newpage

\subsection{テーパーアンプ}
 TAは近赤外の連続波(continuous wave,  cw)のレーザーを$20$ dB以上の利得で増幅することができる素子として知られている\cite{Cruz:06}.\\
 TAは図\ref{TA_structure}のような構造をしており, 半導体レーザーと同じDH構造を有している. TAの特徴は入力側から出力側にだんだんと拡がっているゲイン領域である. また, 増幅器として用いるためへき界面に反射材がコーティングされておらず光フィードバックがないことが半導体レーザーとの大きな違いである.\\
 TAの一つの特徴は従来の細いストライプを持つ半導体レーザーよりも高い輝度を出せることである. 輝度は
 \begin{equation}
   B = P/(A\Omega)
 \end{equation}
で定義される. ただし, Pは光の出力パワー, Aは放出面積, $\Omega$は光の放出される立体角を表す. 輝度は, レーザーの空間モードが一つであるとき, およそ
\begin{equation}
  B = P/\lambda^2
\end{equation}
の最大値をとる\cite{Walpole1996}. ただし, $\lambda$は波長を表す. このようなレーザーのビームを回折限界であるという. 従来の幅の細いゲイン領域ではバルクや発光面の加熱効果によって得られるパワーが数百ミリワットに限られていた. また, より幅のあるゲイン領域を用いると横モードを一つに保つことが難しいという欠点があった.これに対して, cwレーザーをTAを用いて増幅した場合, 数ワットの回折限界の出力を得ることができる\cite{Walpole1996}.


\begin{figure}[htbp]
 \begin{center}
  \includegraphics[width=100mm]{figures/chapter2/TA_structure.png}
 \end{center}
 \caption{TAの基本構造(文献\cite{Walpole1996}から引用)}
 \label{TA_structure}
\end{figure}
\end{comment}
\newpage

\chapter{過去の研究に基づくCs原子の二光子冷却のシミュレーション}
\section{従来の課題と光周波数コムによる冷却のメリット}

 従来のレーザー冷却では, アルカリ金属やアルカリ土類金属などの限られた原子しか冷却できなかった.この理由としては主に2つの理由が挙げられる. 一つ目としては, 水素や酸素を含む多くの原子の遷移エネルギーは真空紫外領域に相当しており現在はこの領域で一光子遷移を通じて冷却を行うための十分な強度のレーザーを得ることができていないことがある. 二つ目は, 多くの原子ではエネルギー準位の構造が複雑であり励起された原子が準安定な準位に緩和してしまうので, これを励起するためのレーザーを用意する必要があり実験の系が複雑化してしまうことである\cite{PhysRevA.73.063407}.\\
 光周波数コムを用いた二光子のレーザー冷却は, 2006年にKielpinski氏らによって提案された\cite{PhysRevA.73.063407}. 光周波数コムを用いることにより, 上記の2つの課題を克服することができる. まず, 光周波数コムは高強度のピークパワーをもつため, 同じ時間平均パワーをもつcwレーザーに比べて高効率の非線形光学効果を利用することができ, より高強度の短波長のレーザーを得ることができる. また, 光周波数コムの複数の縦モードをリポンプレーザーとして利用できるために, 実験の系を簡単にすることができる. これらの長所により, 光周波数コムはcwレーザーよりも効率の良い二光子冷却を実現することができると期待される\cite{PhysRevA.73.063407}.\\
 本章では, コムを用いた二光子冷却についての過去の研究の内容を紹介する.

\section{二光子コムによるレーザー冷却の理論}
 Jayichらの論文\cite{PhysRevX.6.041004}で説明されている, 二光子遷移を用いた光周波数コムレーザー冷却の理論を紹介する.\par
コムによる二光子の遷移を考えるとき, パルスに含まれる二光子のエネルギーもまた, コム(櫛)を形成する.これを二光子コムと呼ぶことにすると, 図\ref{two-photon-comb-figure}のように二光子コムのn番目の縦モードの周波数は
\begin{equation}
f_n = nf_r + 2f_0
\end{equation}
となる. ただし, $f_r$は繰り返し周波数, $f_0$はキャリアエンベロープオフセット周波数を表す. モード同期レーザーについては実効的な共鳴ラビ周波数を求めることができる. 二光子コムのn番目のコムの歯の共鳴ラビ周波数は,
\begin{eqnarray}\label{ResonanceRabi}
\Omega _{n} &=&\sum _{p}\frac {g^{\left( 1\right) }_{p}g^{\left( 2\right) }_{n-p}}{2\Delta_p} \nonumber\\
&=& \sum _ { p } \frac { e ^ { 2 } \mathcal { E } _ { p } \mathcal { E } _ { n - p } } { \hbar ^ { 2 } } \left\langle \mathrm { e } \left| ( \hat { \boldsymbol { \epsilon } } \cdot \mathbf { r } ) \left( \sum _ { \mathrm { i } } \frac { | \mathrm { i } \rangle \langle \mathrm { i } | } { 2 \Delta _ { p } ^ { ( \mathrm { i } ) } } \right) ( \hat { \boldsymbol { \epsilon } } \cdot \mathbf { r } ) \right| \mathrm { g } \right\rangle
\end{eqnarray}\\
ただし, $g^{\left( 1\right) }_{p}$は$p$番目のコムの歯による基底状態から中間状態への共鳴一光子ラビ周波数, $g^{\left( 2\right) }_{p}$は$p$番目のコムの歯による中間状態から励起状態への共鳴一光子ラビ周波数, $\hat { \boldsymbol { \epsilon }}$はレーザーの偏光方向の単位ベクトル, $\mathbf { r }$は電子の位置演算子を表す.
$\Delta _{p}=pf_{r}+f_{0}-f_{gi}$は一光子の中間状態からの離調である. ただし, $f_{gi}$は基底状態から中間状態へのエネルギー差をプランク定数$h$で割ったものである. また, $\epsilon_0$は真空の誘電率, $c$は光速, $e$は電気素量, $\hbar$はプランク定数$h$を$2\pi$で割った値を表す. \\

\begin{figure}[H]
  \centering
    \begin{tabular}{c}
      \begin{minipage}{1\hsize}
        \centering
          \includegraphics[keepaspectratio,  scale=0.2,  angle=0]
                          {figures/chapter3/two-photon-comb-figure.png}
                          \caption{二光子コムの図(文献\cite{PhysRevX.6.041004}より引用. )左の図は一光子コムの全ての縦モードが励起に寄与する様子を示している. 右の図は二光子コムの歯を示している. }
                          \label{two-photon-comb-figure}

      \end{minipage}
    \end{tabular}
\end{figure}

二光子コムのN番目のコムの歯が共鳴周波数に最も近いとき, 速度$v$で動く原子の励起確率の時間平均は,
\begin{equation}\label{ExcitationRate}
\gamma_\mathrm{comb} = \frac{\Omega^2_{N}T_\mathrm{r}} {4} \frac{\sinh(\gamma T_\mathrm{r}/2)}{\cosh(\gamma T_\mathrm{r}/2) - \cos(\delta_N(\bm{v})T_\mathrm{r})}
\end{equation}
ここで, $T_\mathrm{r} \equiv \frac{1}{f_\mathrm{r}}$はパルスの繰り返し時間, $\delta _{N}\left( v\right) \equiv 2\pi ( f_\mathrm{\mu }-f_\mathrm{ge}-f_{N}\widehat {\bm{k}}\cdot {\bm{v}}/c )$は$N$番目の二光子コムの歯の共鳴周波数からの離調を表す. $f_{ge}$は励起状態と基底状態のエネルギー差をプランク定数で割ったもの, $\widehat {\bm{k}}$はレーザーの進行方向の単位ベクトル, $\gamma$は励起準位の自然幅を表す.\\
離調$\delta _{N}\left( v\right)$と自然幅$\gamma$の両方がコムの歯の間隔($2\pi f_r$)よりも十分に小さいとき, 二光子コムは二光子ラビ周波数$\Omega_N$の一つの縦モードとして扱うことができる.この近似の下では, 励起確率は
\begin{equation}\label{EffectiveExcitationRate}
\gamma_N = \frac{\Omega^2_N}{\gamma}\frac{1}{1 + [2\delta_N(\bm{v})/\gamma]^2}
\end{equation}
と表せる.\\
 また, (二色のコムによる冷却ではなく)縮退したコムによる二光子冷却の場合, ドップラー冷却限界温度は
\begin{equation}
  T_\mathrm{D} = \frac{3}{4}\frac{\hbar\gamma}{2k_\mathrm{B}}
\end{equation}
となることが分かっている. ただし, $k_\mathrm{B}$はボルツマン定数である. \\
 Jayichらの論文\cite{PhysRevX.6.041004}では, 初めて光周波数コムを用いた二光子冷却の実証実験が行われた. Jayichらのグループはcwレーザーによる一光子遷移のMOTで捕獲, 予冷されたRb二光子遷移を誘起する光周波数コムを照射することで一次元のレーザー冷却に成功し, MOT中の$146\ {\mu K}$から$57\ \mathrm{\mu K}$への冷却を達成している.

\section{Cs原子の冷却に必要な励起効率の見積もり}
  今回はあらかじめMOTで予冷されたCs原子を冷却することを目指すが, その際に必要な二光子遷移の励起効率を見積もる. $6 ^ { 2 } S _ { 1 / 2 }$の$F = 4$から$6 ^ { 2 } P _ { 3 / 2 }$の$F ^ { \prime } = 5$の遷移によるドップラー冷却を行った場合, ドップラー冷却温度は$125.61\ \mathrm{\mu K}$となる\cite{Cs_level_diagram}. このことから, 二光子コム照射時のCs原子の温度は数百$\mathrm{\mu K}$以下にできると考えられる. 原子の温度が$T$ Kのときの原子の速さの最頻値は
\begin{equation}
  v _ { \mathrm { th } } = \sqrt { \frac { 2 k _ { \mathrm { B } } T } { m }}
\end{equation}
で与えられる. ただし, mは原子の質量を表す. MOT後の原子の速さはこの最頻値を持つものとして計算する. 例えば, $126\ \mathrm{\mu K}$のCs原子の速さの最頻値は$125$ mm/sである. また6s軌道から8s軌道への二光子遷移を利用して冷却する場合, 二つの光子の周波数が同じであるとすると822nm程度の波長を持つ. この場合二光子を吸収した時のCs原子の速度変化は二光子を吸収するごとに$7.3$ mm/s程度の減速である.
ドップラー冷却における加熱効果を無視して, 近似的に原子が等加速度$a\ \mathrm{m/s^2}$で減速するとすると, 速度$v$を持つ原子が停止するまでに移動する距離$l$は
\begin{equation}
    l = \frac{v^2}{2a}
\end{equation}
で与えられる. 実験で用いるレーザーの直径を考慮して原子が冷却されるまでに$0.5$ mm以内の移動距離に抑えるのに必要な加速度から冷却に必要な励起効率を初期温度に応じて計算すると, 図\ref{necessary_excitation_rate}のようになる. この結果と実験上のロスを考慮して今回の実験では$10000\ \mathrm{s^{-1}}$の励起効率を目標とする.
\begin{figure}[htpb]
  \centering
    \begin{tabular}{c}
      \begin{minipage}{1\hsize}
        \centering
          \includegraphics[keepaspectratio,  scale=0.6,  angle=0]
                          {figures/chapter3/necessary_excitation_rate.png}
                          \caption{MOTで予冷したCs原子の冷却に必要な励起効率. 計算の詳細については本文中に記載. }
                          \label{necessary_excitation_rate}

      \end{minipage}
    \end{tabular}
\end{figure}
\section{Cs原子の二色のコムによる励起効率の見積もり}
 Jayichらの論文\cite{PhysRevX.6.041004}では, Rb原子の5s軌道から5d軌道への遷移を利用しており, 以下のパラメータの下で実験を行っている.
\begin{itemize}
  \item 5d順位の線幅 : $2\pi \times 667$ kHz
  \item コムのパルス幅 : $2-5$ ps
  \item コムの平均パワー : $500$ mW
  \item コムビームの直径 : $1$ mm
  \item コムの周波数帯域 : $500$ GHz
  \item コムの繰り返し周波数 : $80$ MHz
\end{itemize}
Jayichらはこのパラメータの下で励起効率を$\gamma_N \sim 13000\ \mathrm{s^{-1}}$と見積もっている. \\
 Jayichらの論文\cite{PhysRevX.6.041004}の励起確率の計算手法に習い, 今回私達の用いるコムでCs原子を冷却する際の励起確率の計算を行った. その計算に際して以下の五つの近似を行った. \\
\\
 (a) $\delta_N(\bm{v}) = 0$とし,
\begin{equation}
  \gamma_N = \frac{\Omega^2_N}{\gamma}
\end{equation}
     とした. \\
 (b) 二光子励起の際の中間状態として$6P_{\frac{3}{2}}$以外の状態を無視した. \\
 (c) 光周波数コムの全ての縦モードの電場の強さが一定であるとして計算を行った.
\begin{equation}
  \mathcal{E}_p = const.
\end{equation}
 (d) 以下の関係式を用いた.
\begin{equation}
  \Sigma_{p} \mathcal{E}_p\mathcal{E}_{n-p} \approx 2I/\epsilon_0 c
\end{equation}
 (e) $\left\langle \mathrm { e } \left| ( \hat { \boldsymbol { \epsilon } } \cdot \mathbf { r } ) | \mathrm { i } \rangle \langle \mathrm { i } |  ( \hat { \boldsymbol { \epsilon } } \cdot \mathbf { r } ) \right| \mathrm { g } \right\rangle$ の値がCs原子とRb原子で等しいとした. \\
\\
 (a)-(d)の近似を用いると, 励起効率の式(\ref{EffectiveExcitationRate})は以下のように計算できる. \\
\begin{eqnarray}\label{approx_ex-rate}
  \gamma_N &=& \frac{\Omega^2_N}{\gamma}\frac{1}{1 + [2\delta_N(\bm{v})/\gamma]^2} \nonumber\\
  &=& \frac{\Omega^2_N}{\gamma}  \nonumber\\
  &=& \frac{1}{\gamma} \Biggl[ \sum _ { p } \frac { e ^ { 2 } \mathcal { E } ^{(1)}_ { p } \mathcal { E }^{(2)} _ { N - p } } { \hbar ^ { 2 } } \left\langle \mathrm { e } \left| ( \hat { \boldsymbol { \epsilon } } \cdot \mathbf { r } ) \left(  \frac { | \mathrm { i } \rangle \langle \mathrm { i } | } { 2 \Delta _ { p } ^ { ( \mathrm { i } ) } } \right) ( \hat { \boldsymbol { \epsilon } } \cdot \mathbf { r } ) \right| \mathrm { g } \right\rangle \Biggr]^2 \nonumber \\
  &=& \frac{1}{\gamma} \Biggl[ \sum _ { p } \frac { e ^ { 2 } \mathcal { E }^{(1)} \mathcal { E } ^ {(2)} } { \hbar ^ { 2 } } \left\langle \mathrm { e } \left| ( \hat { \boldsymbol { \epsilon } } \cdot \mathbf { r } ) \left(  \frac { | \mathrm { i } \rangle \langle \mathrm { i } | } { 2 \Delta _ { p } ^ { ( \mathrm { i } ) } } \right) ( \hat { \boldsymbol { \epsilon } } \cdot \mathbf { r } ) \right| \mathrm { g } \right\rangle \Biggr]^2 \nonumber \\
  &=& \frac{1}{  \gamma   }\Biggl[ \frac{e^2  \mathcal { E } ^ {(1)} \mathcal { E } ^ {(2)}}{2\hbar ^ {2}}\left\langle \mathrm { e } \left| ( \hat { \boldsymbol { \epsilon } } \cdot \mathbf { r } ) | \mathrm { i } \rangle \langle \mathrm { i } |  ( \hat { \boldsymbol { \epsilon } } \cdot \mathbf { r } ) \right| \mathrm { g } \right\rangle\sum _ { p }\frac{1}{ \Delta _ { p } ^ { ( \mathrm { i } ) }} \Biggr]^2\\
  \mathcal{E}^{(i)} &=&  \sqrt{\frac{2 I_{i}}{M \epsilon_0 c}}\ \ \ \ (i = 1,2)
\end{eqnarray}\\
ただし, $\mathcal{E}^{(1)}_p$は基底状態から中間状態へのコムの$p$番目の縦モードの電場の大きさを表し, , $\mathcal{E}^{(2)}_p$は中間状態から励起状態へのコムの$p$番目の縦モードの電場の大きさを表す. $\mathcal{E}^{(i)}\ \ (i = 1,2)$は近似(c)の下でのコムの電場の強さを表す. $I_{i}\ \ (i = 1,2)$はそれぞれのコムの強度を表す. $M$はコムの歯の本数を表し, 計算上では2つのコムの歯の数は等しいとした. \\
 まず, Jayichらのグループが計算で得た励起効率から計算すると
\begin{equation}
\left\langle \mathrm { e } \left| ( \hat { \boldsymbol { \epsilon } } \cdot \mathbf { r } ) | \mathrm { i } \rangle \langle \mathrm { i } |  ( \hat { \boldsymbol { \epsilon } } \cdot \mathbf { r } ) \right| \mathrm { g } \right\rangle = 4.1 \times 10^{-22} \mathrm{m^2}
\end{equation}
を得る. この値をCsでも用いて計算する. \\
 今回の実験ではCsの$6S_{1/2}$から$8S_{1/2}$の二光子遷移を利用して冷却を行う. まず, 中心波長$822$ nmの同じコムから二光子を吸収した場合での励起効率の計算を行った. このときのパラメータを以下に示す.
\begin{itemize}
  \item 遷移 : $6S_{\frac{1}{2}}$から$8S_{\frac{1}{2}}$
  \item 8S順位の線幅 : $2\pi \times 2.18$ MHz
  \item コムビームの直径 : $1$ mm
  \item コムの周波数帯域 : $20$ THz
  \item コムの繰り返し周波数 : $120$ MHz
  \item 中間状態の$6P_{\frac{3}{2}}$への遷移周波数は852nmとする.
\end{itemize}
以上の条件の下で, 式(\ref{approx_ex-rate})を用いて二光子励起効率のコムのパワー依存性を計算すると図\ref{degenerated_2_photon_excitation_rate-P}のようになる. このグラフから, 等実験室のコムのパワーでは励起効率が不足していることが分かる. \\
\begin{figure}[htpb]
  \centering
    \begin{tabular}{c}
      \begin{minipage}{1\hsize}
        \centering
          \includegraphics[keepaspectratio,  scale=0.6,  angle=0]
                          {figures/chapter3/degenerated_2_photon_excitation_rate-P.png}
                          \caption{縮退した二光子で冷却した場合の励起効率. 計算の詳細は本文中に記載. }
                          \label{degenerated_2_photon_excitation_rate-P}
      \end{minipage}
    \end{tabular}
\end{figure}


\section{繰り返し周波数$120$ MHz二色のコムによる冷却に必要な平均パワーの見積もり}
\subsection{一光子遷移による中間状態への励起効率の見積もり}\label{1photon_ex_rate_method}
 上述のシミュレーションで励起効率が低下した要因としては, 中間状態である$6P_{\frac{3}{2}}$からの一光子コムの離調が大きいことが挙げられる. このことから, 周波数の異なる二つの光子による励起により中間状態からの一光子コムの離調を小さくすることで励起効率を向上させることができるのではないかと考えられる. しかし, 二色のコムで冷却する場合は一光子コムの歯と中間状態の離調をどの程度まで小さくできるかについても注意する必要がある. なぜなら, あまりに離調を小さくしすぎると中間状態に電子が励起される一光子励起の過程が支配的になってしまうからである. 離調をどの程度まで小さくできるかについては一光子の散乱効率を評価することで判断した. ビームスポットの直径は$0.5$ mmとし, 二光子冷却時のレーザーの照射時間は, Jayichらの論文での冷却時間を参考にして$5$ msであると仮定する. このレーザーの照射時間における一光子の散乱回数がおよそ$1$以下になることを期待すると, 散乱効率はおよそ$200\ \mathrm{s^{-1}}$以下である必要がある. \\
 散乱効率は次の式で与えられる.
\begin{equation}
  R _ { \text { scatt } } = \frac { \Gamma } { 2 } \frac { \Omega ^ { 2 } / 2 } { \delta ^ { 2 } + \Omega ^ { 2 } / 2 + \Gamma ^ { 2 } / 4 }
\end{equation}
この式は飽和パラメーター$s$を用いて
\begin{equation}
  R _ { \text { scatt } } = \frac { \Gamma } { 2 } \frac{s}{1+s}
\end{equation}
と表せる. $s$は吸収遷移の飽和強度$I_0$と用いるレーザーの強度$I$により,
\begin{equation}
  s = \frac{I/I_0}{1+(2\delta/\Gamma)^2}
\end{equation}
と書けるのでこれを用いて計算する\cite{ノーベル賞と分光学}. なお, 予めMOTで冷却済みのためドップラー効果の寄与は無視した. Cs原子の中間状態は$6P_{\frac{1}{2}}$($\Gamma = 4.56$ MHz,\ $I_0 = 2.50 \mathrm{\ mW/cm^2}$)を用いた\cite{CsDLine}. また, 強度については近似 (c)を用いて, 全体の強度をコムの歯の数で割ったものを一本のコムの歯の強度として, 全てのコムの歯における$R _ { \text { scatt } }$を独立に計算し, その総和をコム全体の$R _ { \text { scatt } }$とした. この際の, コムの周波数幅は$6S_{\frac{1}{2}}$から$6P_{\frac{1}{2}}$
の共鳴周波数付近で$10$ nm程度の波長幅に対応する$5$ THzとした. \par
繰り返し周波数$f_{\mathrm{rep}}$については当実験室にあるコムの周波数である$120$ MHzにおいて計算を行った. その結果を示したのが図\ref{1photon-sc-rate-2dcolor_120MHz_log}である. \\
\begin{figure}[H]
  \centering
    \begin{tabular}{c}
      \begin{minipage}{1\hsize}
        \centering
          \includegraphics[keepaspectratio,  scale=0.35,  angle=0]
                          {figures/chapter3/1photon-sc-rate-2dcolor_120MHz_log.png}
                          \caption{中間状態への励起効率の離調とパワー依存性. 横軸は$6S_{\frac{1}{2}}$から$6P_{\frac{1}{2}}$の共鳴にもっとも近い一光子コムの歯と$6P_{\frac{1}{2}}$の離調を示している. コムの周波数幅は$5$ THz, $f_{\mathrm{rep}}$は$120$ MHzとしている. グラフ中の白い実線が励起効率$200\ \mathrm{s^{-1}}$を表す. }
                          \label{1photon-sc-rate-2dcolor_120MHz_log}
      \end{minipage}\\
  \end{tabular}
\end{figure}

\newpage
\subsection{二色のコムによる冷却に必要なパワーの見積もり}
このように波長の異なるコムを用意するには, 周波数幅の広いコムからバンドパスフィルター(BPF)を用いて目的の周波数帯を取り出す必要がある. しかし, 単に切り出すだけでは光のパワーが低下してしまい励起効率が低下することが予想される. そのため, 二色のコムを用いて励起する場合に二色のどのような強度のコムを用意すると目標の励起効率が得られるのかのシミュレーションを行った. その際前節の議論を踏まえ, $f_{\mathrm{rep}} = 120$ MHzのコムでは, $6P_{\frac{1}{2}}$と最近接の一光子コムの歯の離調が$200$ MHz, $400$ MHzの際の励起効率を二つのコムのパワーに応じて計算した. この際, 一光子の$6P_{\frac{1}{2}}$への散乱効率が$1\ \mathrm{s^{-1}}$よりも大きくならないように注意して, $894$ nm側のコムのパワーを設定した. また, ビームスポットの直径は$0.5$ mmとしている. 計算の結果を図\ref{5THz-120MHz-02GHz_new}, \ref{5THz-120MHz-04GHz_new}に示す. この計算結果から$10000\ \mathrm{s^{-1}}$の励起効率を得るために必要な, 二色のコムのパワーと中間状態からの一光子コムの歯の離調のおよその組み合わせを見積もることができた.

\begin{figure}[H]
  \centering
    \begin{tabular}{c}
      \begin{minipage}{1\hsize}
        \centering
          \includegraphics[keepaspectratio,  scale=0.6,  angle=0]
                          {figures/chapter3/2dcolor/5THz-120MHz-02GHz_new.png}
                          \caption{$f_\mathrm{rep} = 120\ \mathrm{MHz}$で$6P_{\frac{1}{2}}$と最近接の一光子コムの歯の離調が$200$ MHzのときの, 二光子励起効率のパワー依存性. 図中の白い実線が$10000\ \mathrm{s^{-1}}$の励起効率を示す. }
                          \label{5THz-120MHz-02GHz_new}
      \end{minipage}\\
\begin{comment}
    \end{tabular}
\end{figure}
\begin{figure}[H]
  \centering
    \begin{tabular}{c}
\end{comment}
      \begin{minipage}{1\hsize}
          \centering
            \includegraphics[keepaspectratio,  scale=0.6,  angle=0]
                            {figures/chapter3/2dcolor/5THz-120MHz-04GHz_new.png}
                            \caption{$f_\mathrm{rep} = 120\ \mathrm{MHz}$で$6P_{\frac{1}{2}}$と最近接の一光子コムの歯の離調が$400$ MHzのときの, 二光子励起効率のパワー依存性. 図中の白い実線が$10000\ \mathrm{s^{-1}}$の励起効率を示す. }
                            \label{5THz-120MHz-04GHz_new}
        \end{minipage}
    \end{tabular}
\end{figure}

\section{繰り返し周波数$1.6$ GHzの二色のコムによる冷却に必要なパワーの見積もり}
同様の励起効率の見積もりを当研究室にあるもう一種類のコムの繰り返し周波数である$1.6$ GHzにおいても行った.
\subsection{一光子遷移による中間状態への励起効率の見積もり}\label{1.6GHz_1photon}
繰り返し周波数$1.6$ GHzのコムでは繰り返し周波数が大きいため, 図\ref{rityounashi_level_diagram}のように中間準位との共鳴周波数をコムのスペクトルがまたぐ形にし, コムのもつ二つの縦モードの間に中間状態への共鳴周波数が来るようにする. このとき, 中間準位と最近接のコムの縦モードの間に, $0.8$ GHz程度の離調をつけることができる. 上記のようなスペクトルのコムにおける中間状態への一光子励起効率を, \ref{1photon_ex_rate_method}節と同様の手法により見積もった. 見積もりの結果は, 図\ref{1photon-ex-rate-rityounashi}のようになった.

\begin{figure}[H]
  \centering
    \begin{tabular}{c}
      \begin{minipage}{1\hsize}
        \centering
          \includegraphics[keepaspectratio,  scale=0.6,  angle=0]
                          {figures/chapter3/rityounashi_level_diagram.png}
                          \caption{$f_{\mathrm{rep}} = 1.6$ GHzのコムの準位図. $|g\rangle, |i\rangle, |e\rangle$はそれぞれ基底状態, 近共鳴の中間準位, 励起状態を表す. 基底状態と中間状態間の共鳴周波数と, コムの中心周波数が一致している. 中間状態と, 最近接の縦モードの離調は$0.8$ GHzとなっている. }
                          \label{rityounashi_level_diagram}
      \end{minipage}
    \end{tabular}
\end{figure}
\begin{figure}[H]
  \centering
    \begin{tabular}{c}
      \begin{minipage}{1\hsize}
        \centering
          \includegraphics[keepaspectratio,  scale=0.6,  angle=0]
                          {figures/chapter3/1photon-ex-rate-rityounashi.png}
                          \caption{中間状態への励起効率のコムの平均パワーの依存性. $6S_{\frac{1}{2}}$から$6P_{\frac{1}{2}}$の共鳴周波数がコムの中心周波数と一致し, 最近接の一光子コムの歯と$6P_{\frac{1}{2}}$の離調が$0.8$ GHzとした. コムの周波数幅は$5$ THz, $f_{\mathrm{rep}}$は$1.6$ GHzとしている.
                           }
                          \label{1photon-ex-rate-rityounashi}
      \end{minipage}
    \end{tabular}
\end{figure}

\newpage
\subsection{二色のコムによる冷却に必要なパワーの見積もり}
\ref{1.6GHz_1photon}節のように, 二色のコムの中心周波数がそれぞれ, 基底準位と中間準位の共鳴周波数, 中間準位と励起準位の共鳴周波数と一致しているようなスペクトルを考え, このときの二光子励起効率を見積もった.
\begin{figure}[H]
  \centering
    \begin{tabular}{c}
      \begin{minipage}{1\hsize}
        \centering
          \includegraphics[keepaspectratio,  scale=0.6,  angle=0]
                          {figures/chapter3/rityounashi_16GHz.png}
                          \caption{$6S_{\frac{1}{2}}$から$6P_{\frac{1}{2}}$の共鳴周波数がコムの中心周波数と一致し, 最近接の一光子コムの歯と$6P_{\frac{1}{2}}$の離調が$0.8$ GHzとしたときの, 二光子励起効率のパワー依存性. 図中の白い実線が$10000\ \mathrm{s^{-1}}$の励起効率を示す. コムの繰り返し周波数は$1.6$ GHz, 周波数幅は$5$ THzとし, ビームスポットの直径は$0.5$ mmとした. }
                          \label{rityounashi_16GHz}
      \end{minipage}
    \end{tabular}
\end{figure}

\section{まとめ}
Csの典型的な温度のところから冷却をスタートしてその効果を見るためには, 散乱効率が$10000\ \mathrm{s^{-1}}$程度必要である. これを得るために必要な二色のコムのパワーの境界を図中に示した. 周波数コムの光スペクトルをフォトニック結晶ファイバー等で広帯域化した上で, もし当該2波長において増幅を施すことができれば冷却の実証をすることができる. そこで5章では光周波数コムの光増幅に関して実験的に探求を行った.

\chapter{Cs原子のMOTのためのcw光源の構築}
\section{飽和吸収分光法によるCs原子のロック}
\subsection{MOTに用いるCs原子の超微細構造}
 今回のMOTで用いるCs原子の準位図は図\ref{Cs_level_diagram_MOT}のようになっている. 冷却に用いる遷移は$6 ^ { 2 } S _ { 1 / 2 }$の$F = 4$から$6 ^ { 2 } P _ { 3 / 2 }$の$F ^ { \prime } = 5$の遷移であるが, $6 ^ { 2 } S _ { 1 / 2 }$の$F = 3$に脱励起した電子を冷却のサイクルに戻すために$6 ^ { 2 } S _ { 1 / 2 }$の$F = 3$
から$6 ^ { 2 } S _ { 3 / 2 }$の$F ^ { \prime } = 4$の遷移に対応するレーザーも使用する. 便宜的に前者のレーザーをメインレーザー, 後者のレーザーをリポンプレーザーと呼ぶことにする.


\begin{figure}[htpb]
  \centering
    \begin{tabular}{c}
      \begin{minipage}{1\hsize}
        \centering
          \includegraphics[keepaspectratio,  scale=0.35,  angle=0]
                          {figures/saturated-absorption/Cs_level_diagram_MOT.png}
                          \caption{MOTで用いるCs原子の超微細構造の準位図(参考文献\cite{Cs_level_diagram}から引用)}
                          \label{Cs_level_diagram_MOT}
      \end{minipage}
    \end{tabular}
\end{figure}
\subsection{ECDLによる周波数の制御}
今回の実験で飽和吸収分光に用いるためのレーザーの周波数の制御には外部共振器型半導体レーザー(External Cavity Diode Laser, ECDL)を使用している. ECDLは回折格子の持つ波長選択性を用いて特定のモードの1次回折光をレーザーダイオードに戻すことによりそのモードのゲインを上げ, 他のモードのゲインを下げるというものである. この回折格子をピエゾ素子に取り付け, 電気的に回折格子の角度を調整できるようにすることで目的の周波数の周辺でレーザーの周波数の挿引を可能にしている.

\begin{figure}[htpb]
  \centering
    \begin{tabular}{c}
      \begin{minipage}{1\hsize}
        \centering
          \includegraphics[keepaspectratio,  scale=0.35,  angle=0]
                          {figures/saturated-absorption/ECDL_diagram.png}
                          \caption{ECDLの概略図(参考文献\cite{ECDL}より引用)}
                          \label{ECDL_diagram}
      \end{minipage}
    \end{tabular}
\end{figure}
\subsection{FMサイドバンドロック法}
飽和吸収分光で得られたPDの信号の凹みに周波数をロックするために目的の周波数で正負が逆転するような信号(エラー信号)を生成し, フィードバックをかけることで周波数をロックするという手法を用いる. このエラー信号を生成する手法として用いるのがFMサイドバンドロック法である. この手法では, まず周波数$f_\mathrm{0}$のレーザー光の位相に対して周波数$f_\mathrm{m}$のラジオ周波数の変調を加えることで周波数空間上で元の周波数に対して周波数軸上で両側に二本の$f_\mathrm{0} \pm f_\mathrm{m}$のサイドバンドを立てる. 二本のサイドバンドの位相が逆であるため, それぞれのサイドバンドと元の光のヘテロダインビートを取ると, 周波数の違いによるCs原子の吸収の差からエラー信号を生成する.
\subsection{今回の飽和吸収分光に用いる光学系}
 今回の実験では, メインレーザーとリポンプレーザーの飽和吸収分光を同一の気体のCs原子が入ったセルを用いて行った. そのため, 光学系がやや複雑な形となったので概略図をの二つの図に分けて示した. 図\ref{Main_Laser_diagram}はメインレーザーの光学系を示しているが, この光学系の外側に隣接する形で図\ref{repump_diagram}のようにリポンプレーザーの光学系を設置した.
\begin{figure}[htpb]
  \centering
    \begin{tabular}{c}
      \begin{minipage}{1\hsize}
        \centering
          \includegraphics[keepaspectratio,  scale=0.35,  angle=0]
                          {figures/saturated-absorption/Main_Laser_diagram.png}
                          \caption{メインレーザーの光学系}
                          \label{Main_Laser_diagram}
      \end{minipage}\\

      \begin{minipage}{1\hsize}
        \centering
          \includegraphics[keepaspectratio,  scale=0.35,  angle=0]
                          {figures/saturated-absorption/repump_diagram.png}
                          \caption{リポンプレーザーの光学系}
                          \label{repump_diagram}
      \end{minipage}
    \end{tabular}
\end{figure}

\section{測定結果}

メインレーザーの超微細構造を捉えたPDの信号は図\ref{PD_Signal_Main}である. この時のエラーシグナルが図\ref{error_signal_main_all-structure}のようになる. 図\ref{main-locking-error}は周波数ロック時のエラー信号である. これらの測定結果から周波数ロック時の線幅を評価することができる. \\
 PD信号の各極小の位置と超微細構造の対応は図\ref{error_signal_main_all-structure}のようになっている. $F^{\prime} = 3,\ 5$のクロスオーバー共鳴の計測から$F^{\prime} = 5$の計測までの時間$11.5$ msは$226$ MHzに対応しているため, 測定時の周波数の掃引速度は$19.7$ GHz/sとなる. また, 図\ref{error_signal_main_all-structure}を見ると, $F^{\prime} = 5$への遷移に対応する信号の変化の極小から極大への時間幅は$1.2$ msなので$F^{\prime} = 5$
この時間幅に対応する周波数幅は$23$ MHzとなる. また電圧の極大から極小の変化は$V_{\mathrm{pp}} = 0.29$ Vなので, エラー信号の電圧と周波数の対応は$83$ MHz/Vであることが分かる. ここで図\ref{main-locking-error}に示されたロック時のエラー信号の$-0.55$ sから$-0.018$ sまでの部分の標準偏差を求めると$2.4\times 10^{-2}$ Vとなる. これに対応する周波数は$2.0$ MHzであることが分かる. このように, $F^{\prime} = 5$の遷移に幅$2.0$ MHzでロックできていることが分かる. ECDL自体の線幅は通常, 数十から百キロヘルツ程度であるため, 回路の高周波ノイズがのっていると考えられる. \\
 リポンプレーザーについても同様に周波数の掃引とロックを行った. 図\ref{Repump_PD_signal}に実際に観測された超微細構造を示す.

\begin{figure}[htpb]
  \centering
    \begin{tabular}{c}

      \begin{minipage}{1\hsize}
        \centering
          \includegraphics[keepaspectratio,  scale=0.35,  angle=0]
                          {figures/saturated-absorption/PD_Signal_Main.png}
                          \caption{PDで観測されたCs原子の超微細構造(メインレーザー). 光源として用いているECDLの周波数を三角波で掃引しているため, ECDLの周波数が時間に比例して変化している. また, 縦軸の信号は値が小さいほど透過光のパワーが強くなっている. }
                          \label{PD_Signal_Main}
      \end{minipage}\\

      \begin{minipage}{1\hsize}
        \centering
          \includegraphics[keepaspectratio,  scale=0.5,  angle=0]
                          {figures/saturated-absorption/error_signal_main_all-structure.png}
                          \caption{図\ref{PD_Signal_Main}の信号のエラー信号. 横軸については図\ref{PD_Signal_Main}と同様である. }
                          \label{error_signal_main_all-structure}
      \end{minipage}\\

      \begin{minipage}{1\hsize}
        \centering
          \includegraphics[keepaspectratio,  scale=0.25,  angle=0]
                          {figures/saturated-absorption/main-locking-error.png}
                          \caption{メインレーザーの周波数ロック中のエラー信号. $-0.06$ s, $-0.01$ s付近の変化はファンクションジェネレータからの信号によるである. 横軸については図\ref{PD_Signal_Main}と同様である. }
                          \label{main-locking-error}
      \end{minipage}
    \end{tabular}
\end{figure}

\newpage
\begin{figure}[htpb]
  \centering
    \begin{tabular}{c}
      \begin{minipage}{1\hsize}
        \centering
          \includegraphics[keepaspectratio,  scale=0.35,  angle=0]
                          {figures/saturated-absorption/Repump_PD_signal.png}
                          \caption{PDで観測されたCs原子の超微細構造(リポンプレーザー). 光源として用いているECDLの周波数を三角波で掃引しているため, ECDLの周波数が時間に比例して変化している. また, $-0.03$ s, $0.02$ s付近の極小値をとる点において, 掃引信号が折り返している. 縦軸の信号は値が小さいほど透過光のパワーが強くなっている. }
                          \label{Repump_PD_signal}
      \end{minipage}
    \end{tabular}
\end{figure}

\newpage
\chapter{光周波数コムのテーパーアンプによる増幅実験}

\section{光周波数コムのテーパーアンプによる増幅}
 第3章の議論から分かる通り, 二色のコムによるCs原子の冷却を行うに当たり, 数十から数百ミリワットのコムを$761$ nm付近と$895$ nm付近で用意する必要がある. しかし, 現在当研究室が所有する光周波数コムから上記の周波数帯を切り出したときの, 平均パワーは一ミリワット程度しかない. そのため今回の実験では, 光周波数コムから得られた光をテーパーアンプ(Tapered Amplifier, TA)を用いて増幅するという手段を用いる. しかし, 通常TAはcwレーザーを増幅するために用いられるため, 光周波数コムの増幅に用いた場合にどのような振る舞いを見せるかについての過去の研究は限られており, 異なる繰り返し周波数に対してのTAの増幅の振る舞いを調べた研究はまだない. 今回の実験では繰り返し周波数の異なる光周波数コムに対してTAの増幅の様子を測定した.

\section{テーパーアンプの原理}
今回の実験では光周波数コムの光をTAで増幅しているが, この節ではTAの原理について説明する.

\subsection{半導体レーザーの発光原理}
TAは半導体で構成されており, その発光原理は半導体レーザーと同様であり構造も非常に似通っている. このため, まず文献\cite{わかる半導体レーザーの基礎と応用}の議論に沿って, 半導体レーザーの発光原理について説明する. 半導体レーザーの基本構造は図\ref{semicon_structure}のように, 3層のサンドイッチ構造で出来ており, 真ん中に挟まれ発光する層を「活性層」, 活性層を挟み込んで光やキャリアを閉じ込める役割を担う二つの層を「クラッド層」と呼ぶ. このような構造をDH(Double Hetero)構造と呼ぶ.\\
 この半導体p-n接合のエネルギーバンド構造は図\ref{semicon_bands}のようになる. 半導体レーザーでは図\ref{semicon_bands}のように電子が伝導帯から価電子帯に落ちる際のエネルギー差に相当する光が放出されるが, バンド間遷移であるためにライン間遷移に比べ発光スペクトルは広い. この素子に図\ref{semicon_structure}のように順方向のバイアスをかけるとエネルギーバンドが平らになっていき, p-n接合領域に電子とホールが注入され反転分布が形成される. このバンドギャップはGaAsの場合, 約$1.4$ eVで$800$ nmから$900$ nmに相当する\cite{grynberg_aspect_fabre_cohen-tannoudji_2010}. 半導体レーザーでは活性層の両へき界面に鏡をコーティングすることで光フィードバックを実現しレーザー動作を可能としている.

\begin{figure}[htpb]
  \centering
    \begin{tabular}{c}

%----- 半導体レーザーの構造 -----

      \begin{minipage}{0.70\hsize}
        \centering
          \includegraphics[keepaspectratio,  scale=0.35,  angle=0]
                          {figures/chapter2/semicon_structure.png}
                          \caption{半導体レーザーの基本構造}
                          \label{semicon_structure}
      \end{minipage}\\
      \\
      \\

%----- バンド図 -----

      \begin{minipage}{0.70\hsize}
        \centering
          \includegraphics[keepaspectratio,  scale=0.30,  angle=0]
                          {figures/chapter2/semicon_bands.png}
                          \caption{半導体レーザーのバンド図}
                          \label{semicon_bands}
      \end{minipage}

  \end{tabular}
\end{figure}
\newpage

\subsection{テーパーアンプ}
 TAは近赤外の連続波(continuous wave,  cw)のレーザーを$20$ dB以上の利得で増幅することができる素子として知られている\cite{Cruz:06}.\\
 TAは図\ref{TA_structure}のような構造をしており, 半導体レーザーと同じDH構造を有している. TAの特徴は入力側から出力側にだんだんと拡がっているゲイン領域である. ここにマスター光と呼ばれる光を入れることで誘導放出が起き増幅される. また, 増幅器として用いるためへき界面に反射材がコーティングされておらず光フィードバックがないことが半導体レーザーとの大きな違いである.\\
 TAの一つの特徴は従来の細いストライプを持つ半導体レーザーよりも高い輝度を出せることである. 輝度は
 \begin{equation}
   B = P/(A\Omega)
 \end{equation}
で定義される. ただし, $P$は光の出力パワー, $A$は放出面積, $\Omega$は光の放出される立体角を表す. 輝度は, レーザーの空間モードが一つであるとき, およそ
\begin{equation}
  B = P/\lambda^2
\end{equation}
の最大値をとる\cite{Walpole1996}. ただし, $\lambda$は波長を表す. このようなレーザーのビームを回折限界であるという. 従来の幅の細いゲイン領域ではバルクや発光面の加熱効果によって得られるパワーが数百ミリワットに限られていた. また, より幅のあるゲイン領域を用いると横モードを一つに保つことが難しいという欠点があった.これに対して, cwレーザーをTAを用いて増幅した場合, 数ワットの回折限界の出力を得ることができる\cite{Walpole1996}.


\begin{figure}[htbp]
 \begin{center}
  \includegraphics[width=100mm]{figures/chapter2/TA_structure.png}
 \end{center}
 \caption{TAの基本構造(文献\cite{Walpole1996}から引用)}
 \label{TA_structure}
\end{figure}


\newpage
\section{繰り返し周波数$120$ MHzのコムの$766$ nm付近の増幅}
\subsection{測定手法}
繰り返し周波数が$f_{\mathrm{rep}} = 120$ MHzのコムの, $766$ nmを中心波長とする幅$10$ nmのバンドパスフィルター(BPF)を通過した光をTAで増幅させ, マスター光とスレーブ光のパワーを測定した. その際に, BPF通過前のコムのスペクトルと, テーパーアンプの入り口と出口でのコムのスペクトルも測定を行った. その際の光学系は図\ref{760_amp_diagram}に示している. 実験では最初にアイソレータを入れないでTAの実験を行ったところ, 後述の通りTAからの戻り光がコムに光フィードバックをもたらしcw的発振を引き起こすことが観測された. このためアイソレータを使用している.

\begin{figure}[htpb]
  \centering
    \begin{tabular}{c}
      \begin{minipage}{1\hsize}
        \centering
          \includegraphics[keepaspectratio,  scale=0.4,  angle=0]
                          {figures/chapter4/760_amp_diagram.png}
                          \caption{繰り返し周波数$120$ MHzのコムの$766$ nm付近の光を増幅した際の光学系. まずアイソレータなしで測定し, その後アイソレータを設置して測定を行った. }
                          \label{760_amp_diagram}
      \end{minipage}
    \end{tabular}
\end{figure}

\subsection{測定結果}
\subsubsection{TAの戻り光による光周波数コムのスペクトルの変化}
 図\ref{spectrum_current_MODORI}はマスター光入射時にアイソレータを使用しなかった時の, 光周波数コムのキャビティの出力口におけるスペクトラムをTAに印加する電流の大きさを変えつつ分光器で測定したものである. TAの印加電流をあげていくと$770$ nm付近でcw的な発振を起こしていることが分かる. これはTAの入射口から出た自然放出の光が光周波数コムの共振器まで戻り, 光フィードバックを起こしているものと考えられる. \\
 そのためマスター光入射時にアイソレータを通過させたところ, スペクトルは図\ref{spectrum_current_isolator}のようになった. このようなcw的な発振は観測されなかった.
\begin{figure}[H]
 \begin{center}
  \includegraphics[width=140mm]{figures/chapter4/spectrum_current_MODORI.png}
\end{center}
 \caption{戻り光の影響によるコムのスペクトルのTAの印加電流依存性. $f_{\mathrm{rep}} = 120$ MHzのコムで$766$ nm付近の波長を増幅した時のコムの出力口から出た光の一部を測定した. 縦軸は分光器の信号でパワーに比例している.}
 \label{spectrum_current_MODORI}
\end{figure}
\begin{figure}[H]
 \begin{center}
  \includegraphics[width=100mm]{figures/chapter4/comb-spectrum_no-return.png}
\end{center}
 \caption{アイソレータ使用時のコムのスペクトル. $f_{\mathrm{rep}} = 120$ MHzTAのコムで$766$ nm付近の波長を増幅し, 印加電流は$1617$ mAとした. コムの出力口から出た光の一部を計測した. }
 \label{spectrum_current_isolator}
\end{figure}

\subsubsection{TAの増幅の振る舞いの測定結果}
 スレーブ光のパワーのマスター光のパワーによる変化は図\ref{766-M-S-power}のようになった. 出力パワーからマスター光なしのときのパワーを自然放出のパワーとして差し引いた値をスレーブパワーとしている. これ以後の全ての実験でこの処理を行っている. また, その際のスペクトルの変化は図\ref{766_slave-input-power}に示した. また, スレーブ光のパワーとスペクトルのTAの印加電流による変化は図\ref{Gain-current-766}, \ref{766_slave-current_spectrum}のようになった. \\
 図\ref{766-M-S-power}を見るとスレーブパワーは$15$ mW程度で飽和している様子が分かる. また, 図\ref{Gain-current-766}を見ると印加電流に対しては$1400$ mW程度からやや傾きが小さくなっており飽和に近づいていると推測される.
 また, 図\ref{766_slave-input-power}に示されたスレーブ光のスペクトルを見ると, マスターパワーが大きくなると出力のピーク付近の形状がややコムの波長分布を反映してピークが出るが, スペクトルの裾の部分の形状には大きな変化は見られないことがわかる. 次にスレーブ光のスペクトルの電流依存性を示した図\ref{Gain-current-766}を見ると, やはりピーク付近のスペクトルの形状が電流が増えていくにしたがって, コムの形を反映したものになっていることがわかる. 裾の部分の形状に大きな変化は見られない. 図\ref{120-power-comb} - \ref{120-current-766_Master}に測定時のコムとマスター光のスペクトルを示す. なお, 測定を通してスペクトルの形状に大きな変化は見られなかったため, 代表的なものを掲載する.
\newpage
\begin{figure}[htpb]
  \centering
    \begin{tabular}{c}

      \begin{minipage}{1\hsize}
        \centering
          \includegraphics[keepaspectratio,  scale=0.6,  angle=0]
                          {figures/chapter4/766-M-S-power.png}
                          \caption{スレーブ光のパワーのマスター光のパワー依存性. $f_{rep} = 120$ MHzで766nm付近の波長を増幅した. TAには$1600$ mAを印加した. }
                          \label{766-M-S-power}
      \end{minipage}\\
      \begin{minipage}{1\hsize}
        \centering
          \includegraphics[keepaspectratio,  scale=0.7,  angle=0]
                          {figures/chapter4/Gain-current-766.png}
                          \caption{スレーブ光のパワーの印加電流依存性. $f_{rep} = 120$ MHzで766nm付近の波長を増幅した. マスター光のパワーは$7.3$ mW程度を用いた. }
                          \label{Gain-current-766}
      \end{minipage}
   \end{tabular}
\end{figure}

\newpage
\begin{figure}[H]
  \centering
    \begin{tabular}{c}
      \begin{minipage}{1\hsize}
        \centering
          \includegraphics[keepaspectratio,  scale=0.15,  angle=0]
                          {figures/chapter4/766_slave-input-power.png}
                          \caption{スレーブ光のスペクトルのマスター光のパワー依存性.  $f_{rep} = 120$ MHzである. }
                          \label{766_slave-input-power}
      \end{minipage}
  \end{tabular}
\end{figure}

\newpage
\begin{figure}[H]
  \centering
    \begin{tabular}{c}
      \begin{minipage}{1\hsize}
        \centering
          \includegraphics[keepaspectratio,  scale=0.22,  angle=0]
                          {figures/chapter4/766_slave-current_spectrum.png}
                          \caption{スレーブ光のスペクトルの印加電流依存性. $f_{rep} = 120$ MHzである. }
                          \label{766_slave-current_spectrum}
      \end{minipage}
  \end{tabular}
\end{figure}


\newpage

\begin{figure}[H]
  \centering
    \begin{tabular}{c}

      \begin{minipage}{0.5\hsize}
        \centering
          \includegraphics[keepaspectratio,  scale=0.5,  angle=0]
                          {figures/chapter4/120-power-comb.png}
                          \caption{スレーブパワーのマスターパワー依存性測定時のコムのBPF通過前のスペクトル. $f_{rep} = 120$ MHzである. }
                          \label{120-power-comb}
      \end{minipage}
      \begin{minipage}{0.5\hsize}
        \centering
          \includegraphics[keepaspectratio,  scale=0.5,  angle=0]
                          {figures/chapter4/120-power_Master.png}
                          \caption{スレーブパワーのマスターパワー依存性測定時のマスター光のスペクトル. $f_{rep} = 120$ MHzである. }
                          \label{120-power_Master}
      \end{minipage}\\

      \begin{minipage}{0.5\hsize}
        \centering
          \includegraphics[keepaspectratio,  scale=0.5,  angle=0]
                          {figures/chapter4/120-current-comb.png}
                          \caption{スレーブ光のパワーの印加電流依存性測定時のコムのBPF通過前のスペクトル. $f_{rep} = 120$ MHzである. }
                          \label{120-current-comb}
      \end{minipage}
      \begin{minipage}{0.5\hsize}
        \centering
          \includegraphics[keepaspectratio,  scale=0.5,  angle=0]
                          {figures/chapter4/120-current-766_Master.png}
                          \caption{スレーブ光のパワーの印加電流依存性測定時のマスター光のスペクトル. $f_{rep} = 120$ MHzである. }
                          \label{120-current-766_Master}
      \end{minipage}\\
   \end{tabular}
\end{figure}
\newpage

\subsection{考察}
\subsubsection{注入される電子数とレーザーの光子数による考察}
$f_{\mathrm{rep}} = 120$ MHzのコムでは利得の飽和が観測されたが, 二光子冷却の励起効率を考えたときに必要なスレーブ光のパワーが得られていない. そこで, TAによるコムの増幅において利得を上げるためにはどのような手法が必要となるのかを簡単なモデルを用いて原理的に考察を行う. \par
図のような直方体のゲイン領域をもつ半導体の増幅素子を考える. この素子には電流$I$によってキャリアとホールが注入されており, これらの再結合によって光子を生成するメカニズムとなっている. このキャリアの再結合のプロセスにはマスター光による誘導放出に加えて自然放出や, 自然放出の光子による誘導放出など複数のものがあるが, これらを全てまとめた実効的なキャリアの寿命を$\tau$とする. このとき, キャリア密度$n_{\mathrm{c}}$の時間変化を表す微分方程式は
\begin{equation}
  \frac{dn_{\mathrm{c}}}{dt} = \eta \frac{I}{eV}-\frac{n_{\mathrm{c}}}{\tau}
\end{equation}
と書ける. ここで, $\eta$は量子効率, $e_0$は電気素量, $V$はゲイン領域の体積を表す. この微分方程式を初期条件$n_c = 0$の下で解くと
\begin{equation}
  n_c = \frac{\eta\tau I}{e_0V}\left(1-e^{-\frac{t}{\tau}}\right)
\end{equation}
キャリア密度の上限は印加電流の大きさとキャリアの寿命にによって決まることが分かる. キャリア密度の時間変化は図\ref{carrier_saturation}のようになるので, キャリア密度の飽和はキャリアの寿命程度の時間でおこることが分かる. \\
\begin{figure}[H]
 \begin{center}
  \includegraphics[width=100mm]{figures/chapter4/carrier_saturation.png}
\end{center}
 \caption{キャリア密度の時間変化}
 \label{carrier_saturation}
\end{figure}
cw光をTAで増幅した場合, 通常$10$ mW程度のマスター光を$1$ A程度の印加電流で$1$ W程度まで増幅することができる. このとき, $766$ nmのスレーブ光に単位時間あたり含まれる光子数を計算すると,
\begin{equation}\label{TA_photon_rate}
  \frac{1\ \mathrm{W}}{1.6\ \mathrm{eV}\times1.6\times10^{-19}\ \mathrm{C}} = 4\times10^{18}\ 個\mathrm{/s}
\end{equation}
となる. これに対して$1$ Aの電流から毎秒供給される電子数を計算すると,
\begin{equation}
  \frac{I}{e_0} = \frac{1\ \mathrm{A}}{1.6\times10^{-19}\ \mathrm{C}} = 6\times10^{18}\ 個\mathrm{/s}
\end{equation}
となる. 両者を比較するとcw光のTAによる増幅においては注入された電子が効率よく光子に変換できていることが分かる. \par
これに対して, コムのパルスの増幅を行う場合を考える. キャリアの寿命を$500$ psと仮定するとこの時間でTAが供給できるフォトン数は式(\ref{TA_photon_rate})を用いると,
\begin{equation}
  4\times10^{18}\times500\times10^{-12} = 2\times10^9 個
\end{equation}
と計算できる. \par
 これに対して, $f_{\mathrm{rep}} = 120$ MHzのコムが$120$ mWの平均パワーを持つ時の一つのパルスのエネルギーは$10$ pJとなる. 一つのパルスに含まれる光子の数は
\begin{equation}
  \frac{10\ \mathrm{pJ}}{1.5\ \mathrm{eV}\times1.6\times10^{-19}} = 4\times10^7 個
\end{equation}
となる. よってこの計算では増幅の利得は$50$倍程度となる. 実際にはASE(Amplified Stimulated Emission)などの効果でキャリアの寿命はより短くなると考えられる. 以上から, 入力パルスに含まれる光子の数に対して平衡状態にあるキャリアの数が不足していることが, 前節での実験で利得の上限を定めていると考えられる. また, 典型的な半導体のキャリアの寿命は数百psに対して$f_{\mathrm{rep}} = 120$ MHzの場合の繰り返し時間は$T_\mathrm{r} = 8$ ns程度のため, パルス通過後から次のパルスが到達するまでの間に注入されたキャリアの再結合が進んでしまっていると考えられる. これらの考察から, パルス当たりのエネルギーを低下させ, 繰り返し周波数を高めることで利得の向上を見込めると考えた.

\subsubsection{TAによるコムの増幅に関する過去の研究}
過去の研究では$62$ fsの光周波数コムによるパルスの増幅の実験が行われており, この実験でもフェムト秒パルスの利得はcw光と比較すると低くなってしまうことが確認されている\cite{Cruz:06}. この実験ではフェムト秒パルスを$21$ mのシングルモードファイバーを通すことによりパルスの時間幅を$101$ psに伸ばしてTAで増幅したところcw光に近い利得が得られている. また, フェムト秒パルスの利得を低下させる原因として, キャリアの再結合時間$\tau_p$が数百psしかなく, ピークパワーの強い短パルスを照射するとすぐにキャリア密度が減少して利得を使い切ってしまうことが挙げられている. さらに, 超短パルス光を当てるとキャリアが加熱される効果によりTAの飽和エネルギーが低下し, 高いパワーでの利得が減少する効果も挙げられている. \par
この論文の実験結果からも繰り返し周波数を上げ, パルスの一つあたりのエネルギーを下げることで利得を向上させることができるのではないかと考えられる.

\subsubsection{TAのパルスの伝播の数式的なモデル}
レーザーパルスがゲイン領域の内部を群速度$v_\mathrm{g}$程度で伝搬しつつ増幅する様子を表す偏微分方程式は
\begin{equation}
  \left(\frac{\partial}{\partial z} + \frac{1}{v_{\mathrm{g}}}\frac{\partial}{\partial t}\right)P(z,t) = g(z,t)P(z,t)
\end{equation}
と書ける. これに加えて, キャリア密度については
\begin{eqnarray}
  \frac{dn}{dt} &=& \eta \frac{I}{eV} - \frac{n}{\tau(n)}-v_{\mathrm{g}}\sigma(n(z,t))n_{\mathrm{p}}\\
  n_{\mathrm{p}} &=& \frac{P}{\hbar\omega v_{\mathrm{g}}A}
\end{eqnarray}
という方程式を立てることができる. ただし, $n$はキャリア密度, $n_{\mathrm{p}}$は光子密度, $\sigma$は共鳴散乱の効率, $A$はゲイン領域の断面積を表す. 共鳴散乱の効率$\sigma(n(z,t))$については,  
\begin{equation}
  \sigma(n(z,t)) = \sigma_0 \ln{\left( \frac{n(z,t)}{n_0}\right)}
\end{equation}
で与えられる. ここでパラメータ$\sigma_0$は物質固有の定数である. $n_0$は吸収に寄与するキャリアの数である.
\begin{comment}
$\sigma_0$は$1000-5000\ \mathrm{cm^{2}}$程度の値を取ることがわかっている.
\end{comment}
\begin{comment}
$n_0$は$1.5-3\times10^{18}\ \mathrm{cm^{-3}}$程度の値となることが知られている.
\end{comment}
ただし, この微分方程式を差分法で解くと解が発散してしまうため, 数値的に解くことは今後の課題として残っている.

\section{繰り返し周波数$1.6$ GHzのコムの$766$ nm付近の増幅}
\subsection{測定手法}
 前節の考察に基づき, 繰り返し周波数$1.6$ GHzのコムでの増幅実験を行った. \par
$f_{\mathrm{rep}} = 120$ MHzの実験と同様にして, $f_{\mathrm{rep}} = 1.6$ GHzのコムの, $766$ nmを中心波長とする幅$10$ nmのバンドパスフィルター(BPF)を通過した光をTAで増幅させ, マスター光とスレーブ光のパワーを測定した. その際に, BPF通過前のコムのスペクトルと, テーパーアンプの入り口と出口でのコムのスペクトルも測定を行った. その際の光学系は図\ref{astro_amp_diagram_isolator}, \ref{760_astro_amp_diagram}に示している. TAによる増幅のマスターパワー依存性の計測時には図\ref{astro_amp_diagram_isolator}のようにアイソレータを使用した. 一方で, $f_{\mathrm{rep}} = 1.6$
 GHzのコムの実験ではアイソレータを使用しなくてもTAからの戻り光がコムの共振器まで戻らず, cw的な発振を起こさなかったため, TAによる増幅の電流依存性の測定時にはアイソレータは使用しなかった. \\

\begin{figure}[H]
  \centering
    \begin{tabular}{c}
      \begin{minipage}{1\hsize}
        \centering
          \includegraphics[keepaspectratio,  scale=0.35,  angle=0]
          {figures/chapter4/astro_amp_diagram_isolator.png}
          \caption{繰り返し周波数$1.6$ GHzのコムの$766$ nm付近の光を増幅し, TAによる利得の印加電流依存性を測定した際の光学系. }
          \label{astro_amp_diagram_isolator}
      \end{minipage}
    \end{tabular}
\end{figure}
\begin{figure}[H]
  \centering
    \begin{tabular}{c}
      \begin{minipage}{1\hsize}
        \centering
          \includegraphics[keepaspectratio,  scale=0.35,  angle=0]
          {figures/chapter4/760_astro_amp_diagram.png}
          \caption{繰り返し周波数$1.6$ GHzのコムの$766$ nm付近の光を増幅し, スレーブパワーのマスターパワー依存性を測定した際の光学系. }
          \label{760_astro_amp_diagram}
      \end{minipage}
    \end{tabular}
\end{figure}
\newpage
\subsection{測定結果}
 TAによる利得の印加電流依存性については, 図\ref{current-gain_astro766_errorbar}のようなデータが得られた. $900$ mA付近と$1200$ mA付近で利得に滑らかでない揺らぎが見られるが, これは測定時にコムの共振器の状態があまり良くなく, マスターパワーが低下してしまったことが原因だと考えられる. \\
 スレーブパワーのマスターパワー依存性については図\ref{astro_seed_dependency760}に示した. マスターパワーが$0.6$ mW付近から$0.8$ mW付近のスレーブパワーの測定値については揺らぎが見えるが, この理由としてはTAとマスター光ののカップリング効率が時間とともに変化してしまったこと, マスターパワーの変化によりTAのチップの変化が起きてしまったことが考えられる. なお, TAのチップの温度はチップの取り付けられたマウンター上の温度京を通して制御されており, マウンターの温度自体は測定中$19.00\ ^\circ$Cで安定していたが, TAのゲイン領域の温度自体は変化していた可能性が考えられる. 図\ref{760_slave-current_spectrum_astro} - \ref{astro-current-766_master}に測定時のマスター光, スレーブ光とフォトニックファイバー結晶(PCF)通過後のコムのスペクトルを示す. ただし, マスター光とコムのスペクトルについては, 測定を通して大きな変化が見られなかったため, 代表的なものを掲載する.
\begin{figure}[H]
  \centering
    \begin{tabular}{c}
      \begin{minipage}{1\hsize}
        \centering
          \includegraphics[keepaspectratio,  scale=0.7,  angle=0]
                          {figures/chapter4/current-gain_astro766_errorbar.png}
                          \caption{TAによる利得の印加電流依存性. $f_\mathrm{rep} = 1.6$ GHzのコムで$766$ nm付近の光をマスターとして使用した. マスター光のパワーは$100\ \mathrm{\mu W}$から$400\ \mathrm{\mu W}$程度のものを用いた. }
                          \label{current-gain_astro766_errorbar}
      \end{minipage}\\

      \begin{minipage}{1\hsize}
        \centering
          \includegraphics[keepaspectratio,  scale=0.7,  angle=0]
                          {figures/chapter4/astro_seed_dependency760.png}
                          \caption{スレーブパワーのマスターパワー依存性. $f_\mathrm{rep} = 1.6$ GHzのコムで$766$ nm付近の光をマスターとして使用した. TAには$1600$ mAの電流を印加した. }
                          \label{astro_seed_dependency760}
      \end{minipage}

  \end{tabular}
\end{figure}
\newpage
\begin{figure}[H]
  \centering
    \begin{tabular}{c}
      \begin{minipage}{1\hsize}
        \centering
          \includegraphics[keepaspectratio,  scale=0.20,  angle=0]
                          {figures/chapter4/760_slave-current_spectrum_astro.png}
                          \caption{スレーブ光のスペクトルのTAの印加電流依存性. $f_\mathrm{rep} = 1.6$ GHzのコムで$766$ nm付近の光をマスターとして使用した. スペクトルの形状を比較するため, 分光器の信号の最大値が全てのグラフで同じとなるように規格化を行った. }
                          \label{760_slave-current_spectrum_astro}
      \end{minipage}
  \end{tabular}
\end{figure}

\newpage
\begin{figure}[H]
  \centering
    \begin{tabular}{c}
      \begin{minipage}{1\hsize}
        \centering
          \includegraphics[keepaspectratio,  scale=0.15,  angle=0]
                          {figures/chapter4/astro_TA_output_seed_dependency766.png}
                          \caption{スレーブ光のスペクトルのマスター光のパワー依存性. $f_\mathrm{rep} = 1.6$ GHzのコムで$766$ nm付近の光をマスターとして使用した. スペクトルの形状を比較するため, 分光器の信号の最大値が全てのグラフで同じとなるように規格化を行った. }
                          \label{astro_TA_output_seed_dependency766}
      \end{minipage}
  \end{tabular}
\end{figure}

\newpage
\begin{figure}[H]
  \centering
    \begin{tabular}{c}

      \begin{minipage}{0.5\hsize}
        \centering
          \includegraphics[keepaspectratio,  scale=0.5,  angle=0]
                          {figures/chapter4/astro-power-766_Comb.png}
                          \caption{スレーブパワーのマスターパワー依存性測定時のコムのBPF通過前のスペクトル. コムの帯域はフォトニック結晶ファイバーによって拡げられている. $f_{rep} = 1.6$ GHzである. }
                          \label{astro-power-766_Comb}
      \end{minipage}
      \begin{minipage}{0.5\hsize}
        \centering
          \includegraphics[keepaspectratio,  scale=0.5,  angle=0]
                          {figures/chapter4/astro-power-766_master.png}
                          \caption{スレーブパワーのマスターパワー依存性測定時のマスター光のスペクトル. $f_{rep} = 1.6$ GHzである. }
                          \label{astro-power-766_master}
      \end{minipage}\\

      \begin{minipage}{0.5\hsize}
        \centering
          \includegraphics[keepaspectratio,  scale=0.5,  angle=0]
                          {figures/chapter4/astro-current-766_Comb.png}
                          \caption{スレーブ光のパワーの印加電流依存性測定時のコムBPF通過前ののスペクトル. コムの帯域はフォトニック結晶ファイバーによって拡げられている. $f_{rep} = 1.6$ GHzである. }
                          \label{astro-current-766_Comb}
      \end{minipage}
      \begin{minipage}{0.5\hsize}
        \centering
          \includegraphics[keepaspectratio,  scale=0.5,  angle=0]
                          {figures/chapter4/astro-current-766_master.png}
                          \caption{スレーブ光のパワーの印加電流依存性測定時のマスター光のスペクトル. コムの帯域はフォトニック結晶ファイバーによって拡げられている. $f_{rep} = 1.6$ GHzである. }
                          \label{astro-current-766_master}
      \end{minipage}\\
   \end{tabular}
\end{figure}


\newpage
\subsection{繰り返し周波数$120$ MHzのコムの増幅実験との比較}
 スレーブ光強度のマスター光強度依存性を二台のコムで比較すると図\ref{M-S_power-comparison}のようになる. 同じマスター光強度で比較すると, 繰り返し周波数が$1.6$ GHzのコムのスレーブ光強度が上回っていることがわかる. これは同じ光強度の場合, 繰り返し周波数が$120$ MHzのコムでは繰り返し周波数が$1.6$ GHzのコムに比べ一つのパルスに含まれる光子数が多いが, TA内のキャリア数が少ないので誘導放出に使われない光子数が多くなってしまい光が増幅されないことが原因だと考えられる. 一方, $1.6$ GHzのコムでは一パルスあたりの光子数が少ないため, 無駄になる光子が少なく効率よく増幅することができると考えられる. さらに, パルスの時間間隔も$120$ MHzの場合だと$8$ ns程度でキャリアの寿命に対して長すぎるが, $1.6$ GHzの場合$600$ ps程度となるため効率よく誘導放出を起こすことができると考えられる. \\
 図\ref{pulse_power-gain-comparison}は$766$ nm側のコムを増幅した際の利得のパルスエネルギー依存性依存性を二つの繰り返し周波数に応じて比較したものである. 繰り返し周波数$120\ \mathrm{MHz}$の利得をみると, パルスエネルギーの増加に対して利得が低下していく様子がわかる. これは, 各パルスに含まれる光子数に対して励起状態にあるキャリア数が足りておらずパルスエネルギーの増加に対して利得を保てていないことを表していると考えられる. それに対し, 繰り返し周波数$1.6\ \mathrm{GHz}$のコムの利得は$120$ MHzのコムの利得を下回っている. これはパルスの時間間隔が$630$ ps程度で短く, 十分な反転分布が励起されていないことが原因ではないかと考えられる. ただ, $f_\mathrm{rep} = 120$ MHzで測定されたマスター光の最小の平均パワーは$10.5\ \mathrm{\mu W}$であるのに対して, このときのスレーブ光の平均パワーは$0.9$ mWとなっている. スレーブ光の平均パワーに対してマスター光の平均パワーが小さいため, 利得はスレーブ光の平均パワーの測定誤差の影響を受けやすいと考えられる. そのため, パルスエネルギーの小さな領域の利得の比較についてはより測定点を増やした詳細な実験が必要であると考えられる. \\
 また, 二台のコムにおける利得の印加電流依存性を比較すると, 図\ref{TA_cuurent-gain_comparison}のようなグラフが得られる. 繰り返し周波数の高い$f_\mathrm{rep} = 1.6$ GHzのコムは同じ平均パワーの$f_\mathrm{rep} = 120$ MHzよりも高い利得を得ていることが分かる. 前述のような理由からであると考えられる.

\begin{figure}[htpb]
  \centering
    \begin{tabular}{c}

%----- 写真 -----


%----- PD Signal -----

      \begin{minipage}{1\hsize}
        \centering
          \includegraphics[keepaspectratio,  scale=0.7,  angle=0]
                      {figures/chapter4/M-S_power_comparison.png}
                      \caption{二台のコムにおけるスレーブ光パワーのマスター光パワー依存性の比較. TAの印加電流は$1600$ mAである. }
                      \label{M-S_power-comparison}
      \end{minipage}\\

      \begin{minipage}{1\hsize}
        \centering
          \includegraphics[keepaspectratio,  scale=0.23, angle=0]
                          {figures/chapter4/pulse_power-gain-comparison_errorbar.png}
                          \caption{二台のコムにおけるTAの利得のパルスエネルギー依存性の比較. エラーバーは測定時の有効数字の最終桁に起因する不確かさを示す. TAの印加電流は$1600$ mAである. }
                          \label{pulse_power-gain-comparison}
      \end{minipage}
    \end{tabular}
\end{figure}

\newpage
\begin{figure}[H]
  \centering
    \begin{tabular}{c}
      \begin{minipage}{1\hsize}
        \centering
          \includegraphics[keepaspectratio,  scale=0.8,  angle=0]
          {figures/chapter4/TA_cuurent-gain_comparison.png}
          \caption{二台のコムにおけるTAの利得の印加電流依存性の比較. マスター光のパワーは, $f_\mathrm{rep} = 120\ \mathrm{MHz}$のコムでは$7.3$ mW程度のもの, $f_\mathrm{rep} = 1.6\ \mathrm{GHz}$のコムでは$100\ \mathrm{\mu W}$から$400\ \mathrm{\mu W}$程度のものを用いた. }
          \label{TA_cuurent-gain_comparison}
      \end{minipage}
    \end{tabular}
\end{figure}
\newpage
\section{$890$ nm用テーパーアンプのチップマウンターの組み立て}
 今回の実験では, Cs原子のレーザー冷却に必要なパワーを得るためにTAを用いた. 光周波数コムのから$766\ \mathrm{nm}$付近の波長と$890\ \mathrm{nm}$付近の波長を切り出し増幅した. $890\ \mathrm{nm}$側の増幅に用いるTAのチップのマウンターに関しては, 設計と組み立てを行った. TAのチップはeagleyard社のEYP-TPA-0915-01500-3006-CMT03-0000を用いた. TAのチップの構造は図\ref{TA_chip_ds}のようになっている. \\
\begin{figure}[htbp]
 \begin{center}
  \includegraphics[width=70mm]{figures/chapter4/TA_chip_ds.png}
\end{center}
 \caption{TAチップの構造図(eagleyard社のデータシートから引用)}
 \label{TA_chip_ds}
\end{figure}
 TAチップのマウンターの構造は図\ref{TA_mounter_photo_comments}, \ref{TA_mounter_structure}のようになっている. ただし, TAチップの入力光をマスター光, 出力光をスレーブ光と呼ぶ. TAチップは銅のブロックにコリメーションレンズ2枚と共に取り付けられており, そこに温度センサーと直流電源からのSMAケーブルが繋がっている. この銅製のブロックをアルミニウム製のブロックを介して光学定盤に固定している. 二つのブロックの間にペルチェ素子を挟み, 温度を制御している. なお, レンズのマウンターにはアルミニウムを使用している.

\begin{figure}[htpb]
  \centering
    \begin{tabular}{c}

%----- TAチップマウンターの概観 -----

      \begin{minipage}{0.50\hsize}
        \centering
          \includegraphics[keepaspectratio,  scale=0.30,  angle=0]
                          {figures/chapter4/TA_mounter_photo_comments.png}
                          \caption{TAチップマウンターの概観}
                          \label{TA_mounter_photo_comments}
      \end{minipage}
%----- TAチップマウンターの構造図 -----

      \begin{minipage}{0.50\hsize}
        \centering
          \includegraphics[keepaspectratio,  scale=0.35,  angle=0]
                          {figures/chapter4/TA_mounter_structure.png}
                          \caption{TAチップマウンターの構造図}
                          \label{TA_mounter_structure}
      \end{minipage}

    \end{tabular}
\end{figure}
\newpage
 なお, 実際に使用する際には, 図\ref{TA_case}のようにアクリルボードでケースを作り使用した.  また, スレーブ光の形状は光の回折の効果から楕円状になっているため, 垂直方向のコリメーションを銅ブロック状のレンズで行い, 水平方向のコリメーションを追加のシリンドリカルレンズで行う必要がある. \\
\begin{figure}[htbp]
 \begin{center}
  \includegraphics[width=75mm]{figures/chapter4/TA_case.jpg}
\end{center}
 \caption{実際に使用時のTAの様子}
 \label{TA_case}
\end{figure}

\section{繰り返し周波数$1.6$ GHzのコムの$890$ nm付近の増幅}
\subsection{測定手法}
 自作したTAで繰り返し周波数が$1.6$ GHzのコムの, $890$ nmを中心波長とする幅$10$ nmのBPFを通過した光を増幅させた. その際の光学系は図\ref{890_astro_amp_diagram}に示した. マスター光のパワーが一定の下でスレーブ光のパワーの電流依存性の測定を行った. また, TAの印加電流$1.95$ Aの下でスレーブ光のパワーのマスター光のパワーの依存性を測定した. 測定の際にはTAのチップの温度が安定するように, マスター光のパワーが最大の点から測定を開始し, パワーを順に下げていくように測定を行った.

\begin{figure}[H]
  \centering
    \begin{tabular}{c}
      \begin{minipage}{1\hsize}
        \centering
          \includegraphics[keepaspectratio,  scale=0.4,  angle=0]
          {figures/chapter4/890_astro_amp_diagram.png}
          \caption{繰り返し周波数$1.6$ GHzのコムの$890$ nm付近の光を増幅した際の光学系}
          \label{890_astro_amp_diagram}
      \end{minipage}
    \end{tabular}
\end{figure}

\subsection{測定結果}
利得のTAの印加電流依存性については図\ref{TA_power-current_3A_astro}のような測定結果を得た. スレーブパワーが印加電流に対して$1.0$ A程度から線形に増大していき, 飽和は見られなかった. スレーブパワーのマスターパワー依存性については, 図\ref{890TPA_power_dependence_0117}のような結果を得た.

\begin{figure}[H]
  \centering
    \begin{tabular}{c}
      \begin{minipage}{0.50\hsize}
        \centering
          \includegraphics[keepaspectratio,  scale=0.50,  angle=0]
                          {figures/chapter4/TA_power-current_3A.png}
                          \caption{$890$ nm側の繰り返し周波数$1.6$ GHzでのTAのスレーブ光パワーの電流依存性. マスター光のパワーは$1.8$ mW程度を用いた. }
                          \label{TA_power-current_3A_astro}
      \end{minipage}

%----- PD Signal -----

      \begin{minipage}{0.50\hsize}
        \centering
          \includegraphics[keepaspectratio,  scale=0.5,  angle=0]
                          {figures/chapter4/890TPA_power_dependence_0117.png}
                          \caption{$890$ nm側の繰り返し周波数$1.6$ GHzでのTAのスレーブパワーのマスターパワー依存性. TAの印加電流は$1.95$ Aとした. }
                          \label{890TPA_power_dependence_0117}
      \end{minipage} \\

    \end{tabular}
\end{figure}

\begin{figure}[H]
  \centering
    \begin{tabular}{c}
      \begin{minipage}{1\hsize}
        \centering
          \includegraphics[keepaspectratio,  scale=0.5,  angle=0]
          {figures/chapter4/astro-890_comb.png}
          \caption{繰り返し周波数$1.6$ GHzのコムの$890$ nm付近の光を増幅した際の, BPF通過前のコムのスペクトル. スペクトルはフォトニック結晶ファイバーによって拡げられている. }
          \label{astro-890_comb}
      \end{minipage}
    \end{tabular}
\end{figure}

\section{ダブルパスでの増幅実験}

TAは通常ゲイン領域が細い方からマスター光を入射し, ゲイン領域の広い方からスレーブ光を出力するという使用法をする. しかし, 通常の出力側からマスター光を入射し通常の入力側から出た光をミラーで打ち返し再度通常の入力側に光を入射させ, 二度ゲイン領域を通過させ増幅するという手法がある. この光学系の配置をダブルパスといい, これに対して通常の配置をシングルパスと呼ぶことがある. ダブルパスでのcwレーザーのTAの増幅の振る舞いについては先行研究\cite{doi:10.1063/1.3501966}があり, 通常cw光でTAを飽和させるには数十ワットのマスター光が必要となるが, ダブルパスの場合だと$200\ \mathrm{\mu W}$のマスター光で飽和させることが出来ることが分かっている. また, スペクトルの面でもキャリアの, 周波数的に幅の広い自然放出が抑えられることが分かっている. このように, ダブルパスによるメリットは多いが光周波数コムの増幅にダブルパスを用いている研究はまだない. そのため今回の実験では,繰り返し周波数$1.6$ GHzの$761$ nmから$771$ nmのコムをダブルパスにより増幅する実験を行った.


\subsection{測定手法}
 測定時の光学系は図\ref{doublepass_photo}のようになっている. コムの光の直線偏光の角度を半波長板を用いて調整しアイソレーターの出射側にあるPBSを利用して側面から光を入れ, さらに半波長板を用いてテーパーアンプの直線偏光の角度に合わせている. この光をテーパーアンプに, シングルパスとしての使用時の出力口から入力し, シングルパスとしての使用時の入力口から出力させる. この光を跳ね返しミラーで元のパスを逆向きに戻るように反射させると, 光はアイソレータを直進して通過しパワーメータに至る. この光のパワーをパワーメータで計測している. マスター光なしのときの光は, シングルパスとしての使用時の入力口から出てくる自然放出の光を跳ね返しミラーで戻してパワーメータで計測を行っている.
\begin{figure}[H]
 \begin{center}
  \includegraphics[width=120mm]{figures/chapter4/doublepass_photo.png}
\end{center}
 \caption{ダブルパス配置の光学系}
 \label{doublepass_photo}
\end{figure}

\subsection{測定結果}
ダブルパスでの増幅の様子を図\ref{double-pass_I-Gain}, \ref{double-pass_I-Slave}に示す. 図\ref{double-pass_I-Slave}で示されたスレーブ光のスペクトルをみると印加電流が$930$ mA以上ではスレーブ光がcw的に発振してしまっていることが分かる. これは, マスター光が$116\ \mathrm{\mu W}$と小さなパワーしか用意することができなかったため, 使用したTAのゲイン特性の偏りからゲインの高い波長で優先的に自然放出の光が増幅されてしまったものが観測されたと考えられる.

\newpage
\begin{figure}[H]
  \centering
    \begin{tabular}{c}
      \begin{minipage}{0.50\hsize}
        \centering
          \includegraphics[keepaspectratio,  scale=0.4,  angle=0]
                          {figures/chapter4/double-pass_I-slavepower.png}
                          \caption{ダブルパスでのスレーブ光パワーの印加電流依存性. $f_{\mathrm{rep}} = 1.6$ GHzで$890$ nm付近を増幅した. シード光の平均パワーは$112\ \mathrm{\mu W}$である. }
                          \label{double-pass_I-slavepower}
      \end{minipage}
      \begin{minipage}{0.50\hsize}
        \centering
          \includegraphics[keepaspectratio,  scale=0.4,  angle=0]
                          {figures/chapter4/double-pass_I-Gain.png}
                          \caption{ダブルパスでの利得の印加電流依存性. 測定条件は図\ref{double-pass_I-slavepower}と同じである. }
                          \label{double-pass_I-Gain}
      \end{minipage}\\

      \begin{minipage}{1\hsize}
        \centering
          \includegraphics[keepaspectratio,  scale=0.55,  angle=0]
                          {figures/chapter4/double-pass-Slave-Spectrum.png}
                          \caption{ダブルパスでの各印加電流におけるスレーブ光のスペクトル. 測定条件は図\ref{double-pass_I-slavepower}と同じである. }
                          \label{double-pass_I-Slave}
      \end{minipage}
    \end{tabular}
\end{figure}

\begin{figure}[H]
  \centering
    \begin{tabular}{c}
      \begin{minipage}{1\hsize}
        \centering
          \includegraphics[keepaspectratio,  scale=0.5,  angle=0]
          {figures/chapter4/doublepass_master.png}
          \caption{繰り返し周波数$1.6$ GHzのコムでダブルパスで増幅した際のマスター光のスペクトル. }
          \label{doublepass_master}
      \end{minipage}
    \end{tabular}
\end{figure}

\section{結論}

\section{結論}

今回の実験では$766$ nm付近の波長では$35$ mW, $890$ nm付近の波長では$70$ mWのパワーを得た. 図\ref{rityounashi_level_diagram}のように, コムの中心周波数と, 中間準位への共鳴周波数が一致するようなスペクトルを考えると, このパワーのときに近共鳴な中間準位への一光子励起効率は$3\times 10^2\ \mathrm{s^{-1}}$となる. このとき, 実験で得られたパワーでの二光子冷却の励起効率を見積もると$8.2\times10^3\ \mathrm{s^{-1}}$であることが分かり, これは実際に冷却の効果が有意に観測することができる値であると考える. また, 励起効率を二色のパワーによりプロットすると図\ref{result-colorplot}のようになる. \par
なお, 上記の見積もりはビームスポットの直径が$0.5$ mmとして計算を行ったが, 光の強度はビームスポットに反比例するために式(\ref{approx_ex-rate})のように強度の二乗に比例する二光子の励起効率は, ビームスポットの直径の四乗に反比例する. \par
$890$ nm側のコムをこれ以上の平均パワーにすると, 中間準位への一光子励起の効果が無視できないものとなってしまうため, $890$ nm側のコムの平均パワーは今回の増幅で十分な値が得られている. そのため, 今後は$766$ nm側のコムの平均パワーを上げていくことで二光子励起効率の向上を目指したい. ただし, $766$ nmのコムの増幅に用いたTAは今回の実験では$1600$ mAまでしか駆動させなかったが, 最大で$3500$ mAまで印加することができる. このため, 今回の実験系でもTAの印加電流を上げることでまだ増幅率の向上の余地はあると考えられる. また, $890$ nm側のコムについても, $3$ Aまでしか印加していないが, 設計上最大で$4$ Aまで印加することができるため, まだ増幅率の向上が見込める. \par

\begin{figure}[H]
  \centering
    \begin{tabular}{c}
      \begin{minipage}{1\hsize}
        \centering
          \includegraphics[keepaspectratio,  scale=0.5,  angle=0]
          {figures/chapter4/rityounashi_result.png}
          \caption{$6P_{\frac{1}{2}}$と最近接の一光子コムの歯の離調が$700$ MHzのときの二光子励起効率の二色のパワー依存性. 白点線の交点で示されている点が今回得られた二色のパワーの組み合わせである. この点では$8.2\times 10^3\ \mathrm{s^{-1}}$の励起効率が見積もられる. 白実線で示されているのが$10000\ \mathrm{s^{-1}}$の励起効率の曲線である. ビームスポットの直径は$0.5$ mm, 繰り返し周波数は$1.6$ GHz, コムの周波数幅は$5$ THzとした.}
          \label{result-colorplot}
      \end{minipage}
    \end{tabular}
\end{figure}

\chapter{まとめと展望}
\section{まとめ}
本研究では, 現在のcw光では冷却できない原子の光周波数コムによる二光子冷却を目指し, 数値計算と光源開発を主とする基礎技術開発を行った.
\subsection{二光子冷却のための励起効率のシミュレーション}
Cs原子を対象とした, コムによる二光子冷却の励起効率を二色のコムのパワーの関数として見積もりを行った. その際, 共鳴中間状態を$6P_\frac{1}{2}$のみとし, 中間状態に最近接の二光子コムの縦モード以外の影響は無視できるとして励起効率を計算した. その結果, 実験上求められる二光子励起効率として$10000\ \mathrm{s^{-1}}$が必要であることを示した. そして, この励起効率を得るためには異なる周波数を持つ二色のコムによる冷却が有効であることを示した. この二色のコムによる冷却は, 一色のコムによる二光子冷却よりも近共鳴の中間準位を自由に選べるため, より多くの原子の冷却に適用することができる手法でもある. また, 中間状態への一光子遷移を抑えつつ, 目的の二光子励起効率を得るために必要な二色のコムの具体的なパワーの見積もりも行い, 二色のコムによる冷却のためにはコムのパワーの増幅が必要であることを示した. その際, 繰り返し周波数$1.6$ GHzのコムでは, コムのスペクトルが中間準位への共鳴周波数をまたぐことが可能であるため, 繰り返し周波数が$120$ MHzのコムよりも効率の良い二光子励起を起こせることが分かった.
\subsection{TAによるコムの増幅実験}
TAによる増幅実験では, TAを用いたモード同期レーザーの増幅においては繰り返し周波数$120$ MHzのコムよりも繰り返し周波数$1.6$ GHzのコムの方が有効であることを示した. これは, 繰り返し周波数の高周波数化によりパルス当たりの光子数がゲイン領域内の励起されたキャリア数に対して過剰にならないこと, TAのキャリアの寿命と同程度の繰り返し時間となることでキャリアの自然放出による再結合が減り効率的な増幅が可能となることが理由だと考えられる. \par
また, 今回の実験では$766$ nmを中心波長とする波長幅$10$ nmのコムを$35$ mW, $894$ nmを中心波長とする波長幅$10$ nmのコムを$70$ mWに増幅することに成功した. この結果から, コムのスペクトルが中間準位への共鳴周波数をまたぎ, 共鳴周波数と最近接のコムの縦モードの離調を$0.8$ GHzとした際の二光子冷却の励起効率を見積もると$8.2\times10^3\ \mathrm{s^{-1}}$となることが分かった. これは, 冷却実験において冷却の効果が無視できない値であると考えられる. また, 今回の実験では$766$ nmのコムの増幅に用いたTAには$1.6$ A, $890$ nmのコムの増幅に用いたTAには$3$ Aまでしか印加しなかったが, TAの設計上それぞれ$3.5$ A, $4$ Aまで印加することが可能なため, 今回の光学系においてもまだ増幅率の向上が見込めると考えられる. \par

\section{今後の展望}
 目標としていた二光子励起効率を実現するためのパワーを得るために, パワーの増幅率を向上させる必要がある. また, 今回の実験ではMOTで予冷されたCs原子をコムにより二光子冷却するものとして二光子の励起効率を計算したが, 今後の目標として一光子遷移では冷却できない原子を冷却するために, MOTで予冷されていない原子をコムを用いて二光子冷却することが挙げられる. このためにもコムのパワーのさらなる増幅が必要である. そのために考えられる手法としては,
\begin{itemize}
  \item 今回用いたTAによる増幅を複数回行う.
  \item TAに光周波数コムのパルスと同期させたパルス電流を印加することで, 自然放出の光を抑えつつパルス光入射時のキャリア密度を挙げ高い増幅度を得る.
\end{itemize}
などが挙げられる.
そして, 目標とするパワーを得られた場合には, Cs原子に照射する前に二色のコムのフェムト秒パルスのを時間的, 空間的に合わせる作業が必要となる. \par
その結果, Cs原子を用いて二色の光周波数コムによるレーザー冷却を実証することができた場合, 波長変換を通じて短波長のコムを実現して, 従来困難だった原子系や分子系でのレーザー冷却につなげることが可能となるだろう.


\newpage
\chapter*{謝辞}
\addcontentsline{toc}{chapter}{謝辞}
本論文は東京大学工学部物理工学科の卒業論文として書かれたものです. 多くの方からのご指導やお力添えを頂いたことで本論文を完成することができました. この場をお借りして厚く御礼申し上げます. \par
吉岡孝高准教授には, 指導教員として直接指導を頂きました. 回路設計や光学実験について基礎から実際の実験を進め方までを丁寧かつ明快に説明して頂きました. また, 論文の読み方や発表の仕方などについても丁寧に指導して頂きました. 先生の幅広く深い知識や技術を伴った, 明快で本質を抑えた指導や物理に対する姿勢から色々なことを学ばせて頂きました. お忙しい中でも直接の指導を丁寧に行ってくださり, 様々なことを学ばせて頂きました. \par
蔡恩美助教には, 毎日の実験の指導をして頂きました. お忙しい中でも, たくさんの物理や論文の書き方についての質問に細かなところから本当に丁寧に指導して頂きました. また, 日頃の研究室での活動面でも大変お世話になりました. \par
物理工学科工作室の川端光洋さんには, TAのマウンターの作成時の部品を切り出す際に難しい加工をして頂きました.\par
修士二年の池田拓矢さん, 中嶋虹太さんには, 研究室の最高学年として勉強面から生活面まで様々なアドバイスと指導を頂きました. 日頃の実験においても, 実験機器の調整やプログラミングのアドバイスなども頂きました. \par
修士一年の井上真之さん, 山本昂平さんにも, 研究室の先輩として勉強面から生活面まで様々なアドバイスを頂きました. 実験中の疑問に答えて頂いたり, 輪講の場でも指導を頂くなど大変お世話になりました. \par
同期の高村昭吉さん, 田島陽平さんには, 同学年として励ましあったりしながら, 楽しい研究室生活を過ごすことができ感謝しています. \par
東京に送り出し, いつも温く見守ってくださっている両親に非常に感謝しております. \par
この他の方々にも, 多くの支援とご協力を頂きました. 改めて感謝を申し上げます.
\bibliography{reference}
\end{document}
