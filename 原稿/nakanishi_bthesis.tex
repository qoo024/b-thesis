\documentclass[uplatex,dvipdfmx,a4paper,report,papersize,11pt]{jsbook}
\bibliographystyle{jplain}
\title{二色の光周波数コムによるレーザー冷却法の開拓}
\author{物理工学科4年 中西亮}
\date{2018/11/19}
\begin{document}
\maketitle
\newpage

\setcounter{tocdepth}{2}
\tableofcontents


\newpage
\chapter{序論}
\section{研究の背景と目的}
レーザー光の輻射圧を利用して原子の運動を抑制する技術であるレーザー冷却は原子物理において有用な手法として大きな役割を果たしている。1980年以降、研究が本格化したレーザー冷却は現在、ボースアインシュタイン凝縮の実現や光格子時計の精度向上、量子情報処理への利用など様々な応用が研究されている。\cite{レーザー冷却とその応用}\par
しかし、現在のレーザーで冷却することができる原子の種類は非常に限られており、主にアルカリ金属、アルカリ土類金属の原子と希ガス原子の準安定状態である。\cite{PhysRevA.73.063407}この原因として、多くの原子の冷却に必要とされる深紫外領域での高強度のレーザーを現状では実現できていないこと、価電子の多い原子においては電子の遷移サイクルを実現するために多数のリポンプレーザーが必要となり光学系が複雑化してしまうことが挙げられる。\par
この問題を解決する手法の一つとして光周波数コムによる2光子遷移を利用したレーザー冷却の手法が提案されており、実証実験も行われている\cite{PhysRevX.6.041004}\cite{PhysRevA.73.063407}。この論文では光周波数コムの2色の光を利用した2光子遷移により、原子を冷却する手法の開拓を目標としてCs原子の冷却実験を行った。\\



\section{本論文の構成}

\newpage
\chapter{背景知識}

\newpage
\chapter{実験系の構築}







\chapter{実験手法}




\bibliography{reference}
\end{document}
